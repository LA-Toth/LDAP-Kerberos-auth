\section{Előszó}
A számítógépek és az Internet elterjedésével egyre több helyen épülnek ki sokfelhasználós, számos szolgáltatást nyújtó
rendszerek. A legtipikusabb szolgáltatások a bejelentkezést és a levelezést teszik lehetővé.

Kezdetben a felhasználók hitelesítésére elegendő volt egy egyszerű szövegfájl, amely tartalmazta a felhasználó
adatait, valamint a jelszavát is. Mindenkinek joga volt olvasni ezt a fájlt, a jelszavak tárolása ezáltal már nem volt
biztonságos. Ezért a jelszavakat már külön állományban tárolták, amit a gépen egyetlen felhasználó, a root tudott
olvasni, valamint írni.

A biztonsági problémát sikerült elhárítani, azonban két további probléma váltazatlanul fennált, az egyik az, hogy
néhány ezer vagy különösen több tízezer felhasználó esetén kezelhetetlenné vált mind módosítás mind keresés esetén. A
másik akkor merült fel, ha egyazon felhasználónak több gépre is van bejelentkezési joga. Sokszor egy-egy központi
szerveren létrehozzák a felhasználó fiókját, így számos egyéb gépen (például laborokban) azonos körülmények között
folyhat a munka. A felhasználókat tároló fájlt minden gépre át kellene másolni, módosításokat követni, ami nagyon
nehézkes. Ugyanaz a helyzet, mint a gépek azonosításával alig néhány évtizeddel ezelőtt, amikor sok gép volt már, és
az őket azonosító számokat, valamint a könnyebb megjegyezhetőség végett bevezetett beszédes neveket összekapcsoló
állományokat körülményes volt karbantartani. Ezért bevezették a névkiszolgálók rendszerét, s csak egy-egy gépen kell a
módosításokat elvégezni.

A hitelesítésnél ugyanez a megoldás, egy központi gépen tárolható a felhasználó összes adata, jelszava. Többféle
megoldás is szóba kerülhet, például adatbázisban (MySQL, Oracle, stb.), vagy épp egy hierarchikus
rendszerben  (pl. LDAP) történő tárolás. Az LDAP protokoll egy könnyen kezelhető, gyors megoldás nyújt, ezt valósítja
meg az OpenLDAP szerver, valamint a Microsoft Active Directory (AD) vezérlői is. 

Az LDAP adatbázisban is tárolható lenne a jelszó, azonban nem túl biztonságos, ráadásul minden egyes szolgáltatás
igénybevételekor be kellene írni a jelszót. Erre is létezik megoldás, a Kerberos használata, amit az MIT mit-krb5
programja, valamint a Heimdal valósít meg a nyílt forráskódú rendszerek esetén, és a Microsoft AD is tartalmazza.

Az LDAP és a Kerberos segítségével bejelentkezés után többet nem kell beírni a jelszót, azonnal be lehet bejelentkezni
másik gépre is, valamint email-eket is el lehet olvasni. Erről is lesz szó a későbbiekben.

Az LDAP és Kerberos alapú hitelesítést használó programok együttes használatát írtam le, valamint a hozzájuk
kapcsolódó egyéb szolgáltatásokat is, amelyre példa a spam- és vírusszűrés.

Az itt bemutatott rendszer GNU/Linux alapú, amelynek számos terjesztése (disztribúciója) létezik, a beállítások
elsősorban a Gentoo Linux esetén néznek ki így, azonban a többi terjesztésen is hasonlóan oldható meg.

A következő oldalakon az alábbi programok beállítása található meg:
OpenLDAP,
MIT-Krb5 (Kerberos),
Tanúsítvány-kiszolgáló (Certificate Authority),
PAM,
Cyrus saslauthd,
OpenSSH,
Postfix,
Amavisd-new,
ClamAV,
SpamAssassin,
SSL/TLS titkosítás az előző programok esetén
  
A fenti programok beállításait lépésről lépésre nézzük át, hogy végül egy egységesen működő rendszert kapjunk.
  
% Local Variables:
% fill-column: 120
% TeX-master: t
% End:

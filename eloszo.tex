\chapter*{Előszó}
\addcontentsline{toc}{chapter}{Előszó}
A számítógépek és az Internet elterjedésével egyre több helyen épülnek ki sok felhasználós, számos szolgáltatást nyújtó
rendszerek. A legtipikusabb szolgáltatások a bejelentkezést (parancssor kapást) és a levelezést teszik lehetővé.

Kezdetben egy-egy nagy gép volt, amihez terminálokon keresztül csatlakoztak a felhasználók, később egyre gyakoribbá
vált, hogy több számítógéphez is hozzáférhetnek, ámde még elegendő volt a bejelentkezéshez, azonosításhoz szükséges
állományok átmásolása minden gépre. Ma már nem kerülhető el ezek az adatok központi tárolása, ahol néhány kifejezetten
erre a célra fenntartott kiszolgáló számítógép látja el a tárolás feladatát. A felhasználó azonosítása során ezeket a
kiszolgálókat kérdezi meg a rendszer, hogy ismerik-e azt, aki megpróbál bejelentkezni, illetve a helyes jelszót adta-e
meg.

A különböző operációs rendszerek különböző megoldást nyújthatnak erre, azonban az elterjedt megoldások lényegében
azonosak. A helyenként megvalósított hitelesítés Linux esetében lehetséges adatbázisból (pl. MySQL), a gyakori pedig
\textsc{ldap} vagy \textsc{ldap} és Kerberos alapokon nyugszik.

Nagyvállalati környezetben is ez utóbbit használják, méghozzá tisztán Microsoft Windows alapú hálózat esetén Active
Directory alapú tartományvezérlőkkel (\textsc{pdc}, \textsc{bdc}), vegyes környezetben pedig vagy Active Directory és
Services for Unix segítségével egy Windows szerverrel, vagy pedig \textsc{ldap} vagy \textsc{ldap} és Kerberos
használatával egy vagy több linuxos kiszolgálóval.

Ez a jegyzet elsősorban Linuxokon használt  megoldásról szól, bár más \textsc{Unix}-szerű rendszereken is hasonlóan
beállítható. Az \textsc{ldap} és a Kerberos a két legfontosabb terület, azonban a biztonságos használatra és a spam- és
vírusszűréssel ellátott levelezőszerverek beállításáról is szó esik.

A jegyzet viszonylag nagy felhasználói réteget céloz meg, hiszen részben azoknak készült --- és folyamatosan bővül ---,
akik ezzel szeretnének foglalkozni, azonban esetleg még nincsenek teljesen tisztában, hogy mi micsoda, és keresnek egy
magyar nyelvű leírást, valamint azoknak is, akik jól ismerik ezeket a megoldásokat, csak valamelyik részfeladathoz
szükségük van segítségre, támpontra.

\vspace{1em}\noindent Budapest, 2007. március 1.
\begin{flushright}
  \begin{tabular}{@{}ll}
    Tóth László Attila & \textt{panther@elte.hu}
\end{tabular}
\end{flushright}




% Local Variables:
% fill-column: 120
% TeX-master: t
% End:

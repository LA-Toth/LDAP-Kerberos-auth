\chapter{Tanúsítvány-kiszolgáló (CA) beállítása}

Az SSL/TLS titkosítást használó szerverek mindegyike egy-egy tanúsítványt küld a kliensnek. Mire is jó ez? A
kiszolgálót hitelesíti ez, vagyis azt jelzi, hogy a kliens biztosan azzal kommunikál, akivel szeretne. Azaz csak
majdnem. Ugyanazzal a tanúsítvánnyal rendelkező szerverrel.

Alapesetben önaláírtak ezek, vagyis hitelességi értékük csak annyi, hogy elvileg megint ugyanahhoz kapcsolódik a
szerver. No de ha új tanúsítványa lesz a szervernek, akkor máris nem megbízható. Legjobb választás az lenne, ha
egyszerűen nem fogadnánk el ilyet, ha van beleszólásunk.

Van más megoldás is: saját tanúsítvány-kiszolgáló (CA) beállítása, és ennek ún {\em gyökértanúsítványát, root
  certificate-jét} aláiratni egy másik, sokak által ismert tanúsítványkiszolgálóval, vagy csak az adott szerverét
aláiratni. Ez pénzbe kerül. Ha ezt el szeretnénk kerülni, akkor is jó a CA használata, ennek certificate-jét
importálva/felhasználva sok helyen (fentebb már láttunk rá példát), pl böngészőhöz is hozzáadva.

\section{openssl.cnf}
Ez a fájl írja le a CA működését. Több CA is lehet benne, ebből egyik az alapértelmezett (lásd a CA szakasz).

A CA\_default az a CA, amit most használunk, lehetne más neve is, csak így utal arra, hogy ő az alapértelmezett.

\ut{[ ca ]\\
  default\_ca      = CA\_default}

\ut{\begin{tabular}{ll}
    [ CA\_default ] & \\
    dir             &= /root/panthernet.ca\\
    certs           &= \$dir/certs\\
    crl\_dir         &= \$dir/crl\\
    database        &= \$dir/ca.db.index\\
    new\_certs\_dir   &= \$dir/newcerts\\
    &\\
    certificate     &= \$dir/ca/cacert.pem\\
    serial          &= \$dir/ca.db.serial\\
    &\\
    crl             &= \$dir/crl.pem\\
    private\_key    &= \$dir/ca/private/cakey.pem\\
    RANDFILE        &= \$dir/ca.db.rand\\
    &\\
    x509\_extensions &= usr\_cert\\
    &\\
    name\_opt        &= ca\_default\\
    cert\_opt        &= ca\_default\\
    default\_days    &= 365\\
    default\_crl\_days &= 30\\
    default\_md      &= md5\\
    preserve        &= no\\
    policy          &= policy\_match\end{tabular}}

\ut{\begin{tabular}{ll}
    [ policy\_match ] & \\
    countryName             &= match\\
    stateOrProvinceName     &= optional\\
    organizationName        &= match\\
    organizationalUnitName  &= optional\\
    commonName              &= supplied\\
    emailAddress            &= supplied
  \end{tabular}
}

A második lista a CA beállításait tartalmazza, a harmadik pedig azt, hogy a CA mikor tud aláírni egy tanúsítványt:
minek kell egyeznie, mi az, ami nem kötelező, és mi az, ami kötelező, de bármi lehet.

Megadható az is, hogy alapértelmezetten mivel legyen kitöltve a tanúsítvány, valamint hogy milyen kérdések
szerepeljenek, csakúgy, mint az alapértelmezett kulcshosszt is. Ennek részlete:

\ut{\begin{tabular}{ll}
    [\ req\ ]&\\
    default\_bits           &= 1024\\
    distinguished\_name     &= req\_distinguished\_name
  \end{tabular}
}

\ut{\begin{tabular}{ll}
    [\ req\_distinguished\_name\ ]&\\
    countryName                     &= Country Name (2 letter code)\\
    countryName\_default             &= HU\\
    countryName\_min                 &= 2\\
    countryName\_max                 &= 2\\
    stateOrProvinceName             &= State or Province Name (full name)\\
    stateOrProvinceName\_default     &=\\
    localityName                    &= Locality Name (eg, city)\\
    localityMame\_default            &= Budapest\\
    0.organizationName              &= Organization Name (eg, company)\\
    0.organizationName\_default      &= PantherHome\\
    organizationalUnitName          &= Organizational Unit Name (eg, section)\\
    commonName                      &= Common Name (eg, YOUR name)\\
    commonName\_max                  &= 64\\
    emailAddress                    &= Email Address\\
    emailAddress\_max                &= 64\\
  \end{tabular}
}



A CA tanúsítványának létrehozása, 10 évig érvényesen:\\
\texttt{openssl req -config /root/panthernet.ca/openssl.cnf \textbackslash\\
  \beh[2] -new -x509 -keyout /root/panthernet.ca/ca/private/cakey.pem \textbackslash\\
  \beh[2] -out /root/panthernet.ca/ca/cacert.pem  -days 3650}\\

A következő script hoz létre egy ``kérést'' (request) aláírásra, vagyis az aláiratlan  tanúsítványt (.csr: certificate
sign request) és a hozzá tartozó  privát kulcsot (.key fájl).

\ut{\#!/bin/bash\\
  \# file: pnet-careq\\
  if [[ \$\# != 1 ]]; then exit 4; fi\\
  openssl req -config /root/panthernet.ca/openssl.cnf  -nodes \textbackslash\\
  \beh[2] -newkey rsa:2048  -keyout \$1.key -out \$1.csr}

Az előző szkript által generált adatot alá is irathatjuk. A létrehozott, aláírt tanúsítvány a .crt kiterjesztésű
fájlba kerül.

\ut{\#!/bin/bash\\
  \# file: pnet-casign\\
  if [[ \$\# != 1 ]]; then exit 4; fi\\
  openssl ca -config /root/panthernet.ca/openssl.cnf  -policy policy\_match \textbackslash \\
  \beh[2] -out \$1.crt -infiles \$1.csr}

Aláírásra példa:

\ut{zeratul panthernet.certs \# pnet-casign ldaps.panthernet\\\hspace*{1mm}
  Using configuration from /root/panthernet.ca/openssl.cnf\\
  Enter pass phrase for /root/panthernet.ca/ca/private/cakey.pem:\\
  %    DEBUG[load\_index]: unique\_subject = "yes"\\
  Check that the request matches the signature\\
  Signature ok\\
  Certificate Details:\\
  Serial Number: 1 (0x1)\\
  \hspace*{1mm}\ \ \ \ \ \ \ \ Validity\\
  \hspace*{1mm}\ \ \ \ \ \ \ \ \ \ \ \ Not Before: Oct 15 10:36:30 2005 GMT\\
  \hspace*{1mm}\ \ \ \ \ \ \ \ \ \ \ \ Not After : Oct 15 10:36:30 2006 GMT\\
  \hspace*{1mm}\ \ \ \ \ \ \ \ Subject:\\
  \hspace*{1mm}\ \ \ \ \ \ \ \ \ \ \ \ countryName\ \ \ \ \ \ \ \ \ \ \ \ \ \ \ = HU\\
  \hspace*{1mm}\ \ \ \ \ \ \ \ \ \ \ \ organizationName\ \ \ \ \ \ \ \ \ \ = PantherHome\\
  \hspace*{1mm}\ \ \ \ \ \ \ \ \ \ \ \ commonName\ \ \ \ \ \ \ \ \ \ \ \ \ \ \ \ = ldaps.panthernet\\
  \hspace*{1mm}\ \ \ \ \ \ \ \ \ \ \ \ emailAddress\ \ \ \ \ \ \ \ \ \ \ \ \ \ = postmaster@panthernet}


\chapter{Spam- és vírusszűrés}\label{chap:spams-viruses}
Az Amavisd-new egy általános célú levélszűrő, spam- és vírusszűrő programokat hív meg és az általuk adott válasz
alapján eldönti, hogy az adott levél spam-e, vagy épp vírusos-e.

Egy minimális konfiguráció (amit be kell állítani) a következő:

\begin{Verbatim}[frame=single,label=amavisd.conf részlet]
$mydomain = 'panthernet';
$myhostname = 'zeratul.panthernet';
$daemon\_user  = 'amavis';       # (no default;  customary: vscan or amavis)
$daemon\_group = 'clamav';       # (no default;  customary: vscan or amavis)
$virus\_admin = "panther@$mydomain";
$mail\_notify\_admin = "antivirus@$mydomain";
$mail\_notify\_spamadmin = "antivirus@$mydomain";
#### http://www.clamav.net/ 
#$
# ['ClamAV-clamd',
#  \&ask_daemon, ["CONTSCAN {}\n", "/var/run/clamav/clamd.sock"],
#  qr/\bOK\$/, qr/\textbackslash bFOUND$/...
\end{Verbatim}

Az amavisnak a clamav-vel azonos csoportban kell lennie, hogy olvashassa a clamav socket fájlját, a
\texttt{/var/run/clamav/clamd.sock}-t.
A \texttt{mail\_notify*} a levelek feladója, ha értesítést küld a rendszer, hogy spamet vagy mást talált.


Illetve be kell állítani, hogy a könyvtár (\texttt{/var/amavis}) megfelelő tulajdonú legyen:

\ut{\# chown -R amavis:amavis /var/amavis}

\section{A ClamAV beállítása}
A ClamAV alapértelmezett konfigja csak példa, ezért egy hashmark (\#) karaktert kell az ``Example'' sor elé tenni:

\begin{Verbatim}[frame=single,label=clamd.conf részlet]
# Example
LogFile /var/log/clamav/clamd.log
LocalSocket /var/run/clamav/clamd.sock
\end{Verbatim}


A \textt{LocalSocket} ugyanaz legyen, mint ami az \texttt{/etc/amavisd.conf}-ban szerepel. Ha ez megvan, már
működőképes a rendszer.

\section{A Postfix beállítása az Amavisd-New használatára}

A \texttt{/etc/postfix/master.cf} fájlban kell módosítani, az első három sor talán nem kell...


\begin{Verbatim}[frame=single,label=master.cf részlet]
###
# AMAVIS etc
127.0.0.1:10025 inet    n       -       n       -       -       smtpd
  -o content_filter=
  -o local_recipient_maps=
  -o smtpd_client_restrictions=
  -o smtpd_helo_restrictions=
  -o smtpd_sender_restrictions=
  -o smtpd_restrictions=permit_mynetworks,reject_unauth_destination
  -o mynetworks=127.0.0.0/8
  -o strict_rfc821_envelopes=yes
  -o smtpd_error_sleep_time=0
  -o smtpd_soft_error_limit=1001
  -o smtpd_hard_error_limit=1000
smtp-amavis     unix    -       -       n       -       2       smtp
    -o smtp_data_done_timeout=1200
    -o disable_dns_lookups=yes
\end{Verbatim}
\noindent A \texttt{/etc/postfix/master.cf} fájlban is kell egy új sor:

\begin{Verbatim}[frame=single]
content_filter = smtp-amavis:[127.0.0.1]:10024
\end{Verbatim}

Mi is történik ekkor? A Postfix kap egy levelet a 25-ös porton. Ezt továbbadja a localhost 10024-es portjára az
amavisd-new-nak. Az amavisd-new meghívja a spam- és vírusszűrőket, majd visszaküldi a levelet a postfixnak a
localhost 10025-ös portján. Bizonyos feltételek mellett bármikor blokkolódhat a levél. Ha az amavis karanténba zárja,
akkor értesítést küld, alapesetben a localhost 10025-ös portjára.

% Local Variables:
% fill-column: 120
% TeX-master: t
% End:

\section{Postfix}

\subsection{/etc/postfix/main.cf}
A postfix beállításait tartalmazza, például milyen címeket fogadjon el, aliasok hol vannak, levélküldés esetén myilen
gépnevet adjon meg...

\ut{myhostname = zeratul.panthernet\\
  mydomain = panthernet\\
  myorigin = \$myhostname\\
  mydestination = \$myhostname, localhost.\$mydomain, localhost, \$mydomain\\
  relayhost = smtp.axelero.hu\\
  alias\_maps = hash:/etc/mail/aliases,\\
  \beh[2] ldap:/etc/postfix/ldapaliases-mailaliases.cf, \\
  \beh[2] ldap:/etc/postfix/ldapaliases-people.cf}


Alapesetben a \texttt{mydomain} a \texttt{myhostname} végével egyezik meg (nem kötelez megadni). Az SMTP protokollban
ez utóbbi szerepel, vagyis amikor a levelet elküdli a postfix, akkor ilyen néven csatlakozik a másik szerverhez
(\$myhostname). A \texttt{myorigin} az, ami után megadott gépnév fog szerepelni az elkldött levél feladójaként és akár
címzettként is. Például a root küld egy levelet saját magának: \texttt{mail root}. Ekkor mind a feladó, mind a címzett
root@\$myorigin lesz.

A \textt{mydestination} vesszővel elválasztott lista, azon gépneveket tartalmazza, amelyekre fogad levelet, vagyis az
itt felsorolt nevekhez tartzó felhasználók (pl \texttt{user@\$mydomain}) leveleit nem egy másik smtp szervernek
továbbítja, hanem az imap, pop, stb kiszolgálónak, ahonnan az adott felhasználó letöltheti leveleit.

A \texttt{relayhost} az, ahova alapesetben (kivéve a \filename{/etc/postfix/transport} fájlban megadott címzetteket,
tartományokat) továbbítja a leveleket. Többféle formátumban meg lehet adni, IPv6-ot is támogat. Formáutma pl a fenti,
aztán 192.168.0.1:25 is lehet (port számmal). A lehetőségek a konfigurációs fájlban megjegyzésben láthatók.

Amennyiben nincsen \texttt{relayhost}, akkor közvetlenül a címzett levelezőszerveréhez csatlakozik (amit a cím domain
részéhez tartozó MX dns bejegyzésből vesz).

Az \texttt{alias\_maps} mondja meg, hogy egy adott email címhez tartozó levelek hova továbbítódjanak. Itt most az
\filename{aliases} fájlban találhatóak, valamint ldap-ban, a két fájl különböző beállításokat tartalmaz.

\subsection{/etc/mail/aliases}
Debian rendszereken ennek a fájlnak a helye: \filename{/etc/aliases}.

Formátumra részlet:\\
\ut{webmaster:  root\\
  www:                webmaster\\
  \\
  levlista: panther@elte.hu root}

Jelen esetben a www címre küldött levelek a webmasternek továbbítódnak, ami még mindig csak virtulális cím, vagyis a
neki címzett levelek a root felhasználónak továbbítódnak. Így a www címre küldött leveleket is a root kapja meg.

Kisebb, statikus levelezési listák is megadhatóak így, ahol több címzett van.

A Postfix ezt még nem kezeli, ezért adatbázist kell létrehozni belőle (erre utal a \texttt{hash} a fájlnév előtt), ez
a \texttt{/usr/bin/newaliases} paranccsal tehető meg. A létrehozott fájl: /etc/mail/aliases.db.



\subsection{/etc/postfix/ldapaliases.cf}
Az ldap-ban történő keresést az alábbi szerkezetű fájl teszi lehetővé.

\ut{server\_host = ldap://ldaps.panthernet:389\\
  search\_base = ou=MailAliases,dc=panthernet\\
  version = 3\\
  scope = one\\
  bind = no\\
  query\_filter = (maillocaladdress=\%s)\\
  result\_attribute = mailroutingaddress\\
  dereference = 3\\
  timeout = 10}

Az első sor azt az URI-t adja meg, ami a szervert azonosítja. A postfix a \texttt{search\_base} beállításban megadott
csúcs alatti részfában keres. Mivel a \texttt{scope} értéke ``one'', az előbb megadott csúcs gyerekeit nézi végig. A
3-as verzió  a mostani. A \texttt{bind = no} a névtelen kapcsolódást teszi lehetővé (mint a /etc/ldap.conf-ban a
hasonló beállítás).

% Local Variables:
% fill-column: 120
% TeX-master: t
% End:

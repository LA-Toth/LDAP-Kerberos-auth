\chapter{Levéltovábbítás és levélfogadás}\label{chap:Postfix}

A legtöbb esetben egy-egy szerveren megtalálható egy kisebb-nagyobb tudású levelező szerver, pontosabban levéltovábbító
szerver (MTA, Mail Transfer Agent), amely a címzettől függően küldi tovább egy másik kiszolgálónak a leveleket. Néha
nincsen erre szükség, ugyanis a kliens programnál beállítható, hogy egy adott MTA kapja meg az elküldött leveleket (és
ezen a másik gépen menjen keresztül a teljes levelezés). Egy ilyen helyzet lehet az, amikor a felhasználók által
használt gép csupán bejelentkezésre és parancssori munkára szolgál (login server).

Azonban a felhasználóknak meg is kell kapniuk a leveleket, ezért egy vagy több kiszolgálóra komolyabb levéltovábbító
szervert tesznek, a legismertebbek a Postfix, az Exim és a Sendmail. Mi most csak az előbbivel foglalkozunk, elsősorban az
LDAP és Kerberos felhasználásával -- és \aref{chap:spams-viruses}. fejezetben a spam- és vírusszűréssel -- foglalkozunk.

\section{Postfix}
A nagyobb programok meglehetősen bonyolultakká, összetetté fejlődtek az idők folyamán, a Postfix mégis egyszerű
tervezésű és könnyebben konfigurálható maradt. Nagyon sok minden beállítható, egyszerűsége nem is ebben, hanem
modularitásában rejlik, számos kisebb programból áll (\cite{szerverved}).

A Postfix a leveleket több várakozási sorban kezeli, egészen pontosan négyben, és nem csak egyetlen nagy várakozási sort
használ.

Ezek a következők:

\begin{itemize}
  \item Levélgyűjtő sor (Maildrop queue)\\
    A helyileg fogadott levelek kerülnek ebbe a sorba, melyből formázás és esetleges javítás után a beérkező sorba
    jutnak.
  \item  Beérkező sor (Incoming queue)\\
    A helyileg fogadott leveleken felül a más kiszolgálókról a postfix \texttt{smtpd} folyamatán keresztül érkezeett
    levelek kerülnek ide, amíg nincs üresedés az aktív sorban.
  \item Aktív sor (Active queue)\\
    Egy rövid sor, melybe csak akkor érkezik levél, ha van szabad hely. A Postfix ebből próbálja kézbesíteni a
    leveleket.
  \item Halasztott sor (Deferred queue)\\
    Azokat a leveleket tartalmazza, melyeket nem sikerült kézbesíteni, így a rendszer nem próbálja újból és újból
    továbbküldeni. Ha egy tartomány nem elérhető, akkor az összes oda szóló levél egy valamekkora várakozási ideig a
    halasztott sorba kerül. Ha a várakozási idő letelik, akkor az aktív sorba kerül. Ha még mindig nem sikerült
    kézbesíteni, akkor egy új, hosszabb várakozási idővel visszakerül a halasztott sorba.
\end{itemize}



\section{/etc/postfix/main.cf}
A postfix beállításait tartalmazza, például milyen címeket fogadjon el, aliasok hol vannak, levélküldés esetén milyen
gépnevet adjon meg...

\begin{Verbatim}[frame=single]
myhostname = zeratul.panthernet
mydomain = panthernet
myorigin = $myhostname
mydestination = $myhostname, localhost.$mydomain, localhost, $mydomain
relayhost = smtp.axelero.hu
alias_maps = hash:/etc/mail/aliases,
  ldap:/etc/postfix/ldapaliases-mailaliases.cf,
  ldap:/etc/postfix/ldapaliases-people.cf
\end{Verbatim}

Alapesetben a \texttt{mydomain} a \texttt{myhostname} végével egyezik meg (nem kötelez megadni). Az SMTP protokollban
ez utóbbi szerepel, vagyis amikor a levelet elküdli a postfix, akkor ilyen néven csatlakozik a másik szerverhez
(\$myhostname). A \texttt{myorigin} az, ami után megadott gépnév fog szerepelni az elkldött levél feladójaként és akár
címzettként is. Például a root küld egy levelet saját magának: \texttt{mail root}. Ekkor mind a feladó, mind a címzett
root@\$myorigin lesz.

A \texttt{mydestination} vesszővel elválasztott lista, azon gépneveket tartalmazza, amelyekre fogad levelet, vagyis az
itt felsorolt nevekhez tartzó felhasználók (pl \texttt{user@\$mydomain}) leveleit nem egy másik smtp szervernek
továbbítja, hanem az imap, pop, stb kiszolgálónak, ahonnan az adott felhasználó letöltheti leveleit.

A \texttt{relayhost} az, ahova alapesetben (kivéve a \filename{/etc/postfix/transport} fájlban megadott címzetteket,
tartományokat) továbbítja a leveleket. Többféle formátumban meg lehet adni, IPv6-ot is támogat. Formáutma pl a fenti,
aztán 192.168.0.1:25 is lehet (port számmal). A lehetőségek a konfigurációs fájlban megjegyzésben láthatók.

Amennyiben nincsen \texttt{relayhost}, akkor közvetlenül a címzett levelezőszerveréhez csatlakozik (amit a cím domain
részéhez tartozó MX dns bejegyzésből vesz).

Az \texttt{alias\_maps} mondja meg, hogy egy adott email címhez tartozó levelek hova továbbítódjanak. Itt most az
\filename{aliases} fájlban találhatóak, valamint ldap-ban, a két fájl különböző beállításokat tartalmaz.

\section{/etc/mail/aliases}
Debian rendszereken ennek a fájlnak a helye: \filename{/etc/aliases}.

Formátumra részlet:\\
\begin{Verbatim}[frame=single,label=aliases részlet]
www:                webmaster

levlista: root user@domain1.com  user2@example.com
\end{Verbatim}

Jelen esetben a \texttt{www} címre küldött levelek a webmasternek továbbítódnak, ami még mindig csak virtulális cím,
vagyis a neki címzett levelek a \texttt{root} felhasználónak továbbítódnak. Így a \texttt{www} címre küldött leveleket
is a \texttt{root} kapja meg.

Kisebb, statikus levelezési listák is megadhatóak így, ahol több címzett van (pl. a \texttt{levlista});

A Postfix ezt még nem kezeli, ezért adatbázist kell létrehozni belőle (erre utal a \texttt{hash} a fájlnév előtt), ez
a \texttt{/usr/bin/newaliases} paranccsal tehető meg. A létrehozott fájl: /etc/mail/aliases.db.



\section{LDAP-ban tárolt alias-ok}
Az ldap-ban történő keresést az alábbi szerkezetű fájl teszi lehetővé.


\begin{Verbatim}[frame=single,label=/etc/postfix/ldapaliases.cf]
server_host = ldap://ldaps.panthernet:389
search_base = ou=MailAliases,dc=panthernet
version = 3
scope = one
bind = no
query_filter = (maillocaladdress=%s)
result_attribute = mailroutingaddress
dereference = 3
timeout = 10
\end{Verbatim}
Az első sor azt az URI-t adja meg, ami a szervert azonosítja. A postfix a \texttt{search\_base} beállításban megadott
csúcs alatti részfában keres. Mivel a \texttt{scope} értéke ``one'', az előbb megadott csúcs gyerekeit nézi végig. A
3-as verzió  a mostani. A \texttt{bind = no} a névtelen kapcsolódást teszi lehetővé (mint a /etc/ldap.conf-ban a
hasonló beállítás).

\section{Virtuális tartományok}
Gyakran előfordul, hogy egy-egy szerveren egy vagy több tényleges tartomány (domain)van, amelyek levelei a tényleges
felhasználóknak jönnek, azonban ezek mellett még további virtuális tartományok vannak. A virtuális tartományokra az a
jellemző, hogy az oda érkező levelek továbbításra kerülnek. Postfix esetén külön fel kell sorolni a virtuális
tartományokat, továbbá azt is meg kell adni, hogy mi legyen a levelekkel. A legegyszerűbb esetben -- ezt nézzük meg most
-- a levelek továbbítását kell megadni. A \texttt{main.cf}-ben megadjuk, hogy részben LDAP-ból, részben pedig
hagyományos módon egy fájlból (hash prefix) történjen a cím-címzett megfeleltetés.

\begin{Verbatim}[frame=single,label=main.cf részlet]
virtual_alias_domains = mytest.hu, mytest2.hu
virtual_alias_maps = hash:/etc/postfix/virtual,
     ldap:/etc/postfix/virtual-ldap-domains.conf
\end{Verbatim}

Az LDAP változatnál annyi a változataás, hogy nem az \texttt{ou=MailAliases} alatt szerepelnek ezek, hanem önálló
szervezeti egységben (\texttt{ou=VirtualDomains}) , és szintén szervezeti egységnek tekintjük az egyes tartományokat,
vagyis részfában fog keresni a postfix. Ezért a \texttt{scope = sub} lesz a helyes beállítás a konfigurációs fájlban.

Ha egyszerű fájlt használunk, akkor mindig ki kell adni a \texttt{postmap /etc/postfix/virtual} parancsot, hogy
frissüljön az az adatbázisfájl is, amit a postfix olvas (\texttt{/etc/postfix/virtual.db}).

Amennyiben azt szeretnénk, hogy minden egyéb levél, aminek az email címe nincs felsorolva egy adott címre vagy
felhasználónak menjen, akkor a \texttt{@domain1.hu} stílusú (felhasználó nélkül megadott) email címre kell megmondani,
mi történjen vele.

\begin{Verbatim}[frame=single,label=virtual]
# lokális címre küldés (tényleges felhasználó)
user1@mytest.hu panther
user2@mytest.hu root

# másik tartományba küldés
user3@mytest.hu azon@example.com

# minden egyéb, az adott tartományba érkező levél küldése
# angolul: ``catch-all''
@mytest.hu  postmaster@example.net
\end{Verbatim}

\begin{Verbatim}[frame=single,label=virtual-ldap-domains.conf]
server_host = ldap://ldaps.panthernet:389
search_base = ou=VirtualDomains,dc=panthernet
version = 3
scope = sub
bind = no
query_filter = (maillocaladdress=%s)
result_attribute = mailroutingaddress
dereference = 3
timeout = 10
\end{Verbatim}


Ilyen körülmények között felmerül a kérdés, hogy pontosan mi lesz a levelekkel? A problémák, kérdések:

\begin{enumzjr}
  \item \emph{Egy címhez több különböző cél adott.} Ha pl. a user1@mytest.hu ldap-beli célja (mailRoutingAddress
  bejegyzése) más, pl. user4@example.net, akkor mi történik?
  \item \emph{Egy címhez több különböző helyen definiált cél adott.} Pl. a uer1@mytest.hu célja az ldapban ugyanaz, mint
  a fájlban, akkor mi történik?
  \item \emph{A fájlban adott catch-all cím mellett vannak további bejegyzések az ldap-ban is.}
  \item \emph{Ldap-ban definiált catch-all cím.}
\end{enumzjr}

A válaszok:

\begin{enumzjr}
  \item A \texttt{virtual\_alias\_maps}-ban megadott sorrend számít, ezért amelyik előrébb van, az dönt.
  \item Az előzővel azonos.
  \item A catch-all cím (alapértelmezett cím) mindig az utolsó, ezért a postfix előbb végignézi az összes lehetséges
    bejegyzést, hátha talál olyat, ami erre vonatkozik, ha nincs, csak utána foglalkozik az alapértelmezett címmel.
  \item A catchall cím mindig az utolsó, akárhol is definiált.
\end{enumzjr}


\subsubsection{Az LDAP bejegyzések}

Az email címek csoportosíthatóak a következők szerint: \texttt{ou=VirtualDomains} szervezeti egység alatt van minden
ilyen bejegyzés. Ez alatt szervezeti egységként az összes tartomány. Ezek alatt pedig az email címek, amelyek nyilván
egyediek, ezért az \texttt{account} objektumosztály megfelelő erre, akárcsak a \texttt{MailAliases} szervezeti
egységben, és be kell állítani az \texttt{uid} attribútumot, amely lehet a cím \texttt{@} előtti része.

Most álljon itt néhány példa, ilyen sorrendben is kell létrehozni őket:


%% \begin{figure}
%% \caption{A virtuális tartományok és azok bejegyzései}
\begin{Verbatim}[frame=single,label=VirtualDomains]
dn: ou=VirtualDomains,dc=panthernet
ou: VirtualDomains
objectClass: top
objectClass: organizationalUnit
description: Postfix Virtual Domains Root node
\end{Verbatim}


\begin{Verbatim}[frame=single,label=mytest.hu]
dn: ou=mytest.hu,ou=VirtualDomains,dc=panthernet
ou: mytest.hu
objectClass: top
objectClass: organizationalUnit
\end{Verbatim}


\begin{Verbatim}[frame=single,label=mytest2.hu]
dn: ou=mytest2.hu,ou=VirtualDomains,dc=panthernet
ou: mytest2.hu
objectClass: top
objectClass: organizationalUnit

\end{Verbatim}

\begin{Verbatim}[frame=single,label=usr@mytest2.hu]
dn: uid=usr,ou=Mytest2.hu,ou=VirtualDomains,dc=panthernet
objectClass: account
objectClass: top
objectClass: inetLocalMailRecipient
mailLocalAddress: usr@mytest2.hu
uid: usr
mailRoutingAddress: panther
\end{Verbatim}

\begin{Verbatim}[frame=single,label=default]
dn: uid=default,ou=Mytest2.hu,ou=VirtualDomains,dc=panthernet
objectClass: account
objectClass: top
objectClass: inetLocalMailRecipient
mailLocalAddress: @mytest2.hu
uid: default
mailRoutingAddress: panther@example.net
\end{Verbatim}
%% \end{figure}

Azonban nem kötelező az ilyen elrendezés, használható akár maga az email cím is az \texttt{uid} attribútum
értékeként. Itt az is változtatás, hogy közvetlenül a \texttt{VirtualDomains} alatt van, amely azonban átláthatatlanná
tehet egy bonyolultabb adatbázist (fastruktúrát).

%% \begin{figure}
%% \caption{Email cím mint egyedi név}
\begin{Verbatim}[frame=single,label=user1@mytest.hu]
dn: uid=user1@mytest.hu,ou=VirtualDomains,dc=panthernet
objectClass: account
objectClass: top
objectClass: inetLocalMailRecipient
mailLocalAddress: user1@mytest.hu
mailRoutingAddress: user@example.net
uid: user1@mytest.hu
\end{Verbatim}
%% \end{figure}

% Local Variables:
% fill-column: 120
% TeX-master: t
% End:

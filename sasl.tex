\chapter{SSL, OpenLDAP, SASL beállítása}
Az OpenLDAP is támogatja a Kerberost, ehhez neki is meg kell adni egy keytab fájlt. A szolgáltatásnév
\texttt{ldap/gepnev}. Például:

\ut{kadmin.local:  \emph{\textbf{ktadd -k /etc/krb5.keytab.ldap ldap/zeratul.panthernet}}\\
  Entry for principal ldap/zeratul.panthernet with kvno 3, encryption type Triple DES cbc mode with HMAC/sha1 added to keytab WRFILE:/etc/krb5.keytab.ldap.\\
  Entry for principal ldap/zeratul.panthernet with kvno 3, encryption type DES cbc mode with CRC-32 added to keytab
  WRFILE:/etc/krb5.keytab.ldap.}

\noindent A \texttt{/etc/openldap/slapd.conf} fájlt be kell állítani a SASL (GSSAPI) használatára:

\ut{sasl-realm panthernet\\
  sasl-regexp uid=([\^,]+),.*cn=GSSAPI uid=\$1,ou=People,dc=panthernet\\
  sasl-host zeratul.panthernet
}

\noindent Az init scriptben meg kell adni, hol van a keytab fájl, még a slapd indulása előtt:

\ut{export KRB5\_KTNAME=/etc/krb5.keytab.ldap}

\noindent Kerberos ticket nélkül:

\ut{panther@zeratul ~ \$ \emph{\textbf{ldapsearch -Y GSSAPI  uid=panther}}\\
  SASL/GSSAPI authentication started\\
  ldap\_sasl\_interactive\_bind\_s: Local error (-2)\\
  additional info: SASL(-1): generic failure: GSSAPI Error: Miscellaneous failure (No credentials cache found)\\
  panther@zeratul ~ \$ \emph{\textbf{ldapsearch   uid=panther}}\\
  ldap\_sasl\_interactive\_bind\_s: No such object (32)
}


Azzal együtt már működik:
\ut{\$ \emph{\textbf{ldapsearch -Y GSSAPI  uid=panther}}}


% Local Variables:
% fill-column: 120
% TeX-master: t
% End:

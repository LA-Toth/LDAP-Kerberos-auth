\chapter{Az \textsc{ldap} és az OpenLDAP kiszolgáló}

A jegyzetben ismertetett rendszer egyik központi része az OpenLDAP szerver. A kiszolgáló az \textsc{ldap} protokollt
({\em Lightweight Directory Access Protocol}) használja, amely egy ``könyvtárszolgáltatás'' (directory service) elérését
teszi lehetővé, és alapvető jellemzője, hogy olvasásra, keresésre, böngészésre (átnézésre)
optimalizált adatbázist használ az adatok tárolására. Az adatbázis lényegében \emph{bármilyen adatot} tartalmazhat
\emph{bármiről}, az egyetlen megkötés, hogy faszerkezetű hierarchiába kell szervezni.

Az OpenLDAP szerverhez általában \textsc{tcp}/\textsc{ip} kapcsolaton keresztül lehet csatlakozni, azonban a helyi gépen (amelyen az
\textsc{ldap} kiszolgáló található), unix socketen keresztül is történhet a kapcsolódás.

A következőkben az \textsc{ldap} szerver alapvető beállítását, majd pedig a rendszer kibővítését titkosítással,
Kerberos alapú hitelesítéssel, végül pedig a szerver adatbázisának replikálását olvashatjuk.

\section{Mi is az \textsc{ldap}?}

A legelső kérdés, hogy mi az \textsc{ldap}?
Az \textsc{ldap} egy hierarchikusan (faszerkezetbe) szervezett adatbázis elérésére szolgáló protokoll. A fában tárolt
adatokat
egy-egy egyedi név: {\em Distinguished Name (DN)} azonosítja, amely név több részből áll, a  különböző részek alapján
történik a bejegyzések (objektumok) fába szervezése. Minden bejegyzéshez tartozik egy vagy több
tulajdonság, amit egy {\em kulcs-érték} pár határoz meg. A kulcs lehet például a ``common name'' (cn) - mindennapi
név, a hozzá tartozó érték meg ``Gipsz Jakab'', vagy épp a levélcím megadására szolgáló ``mail'' mező. Egy-egy
tulajdonság (lényegében maga a~kulcs) többször is szerepelhet egy objektumon belül.

A fa csúcsainak elhagyható és kötelező tulajdonságait az objektumok (csúcsok) osztályai írják le
(tulajdonságosztályok). A nevük: \texttt{objectclass}, és a jelölés erejéig legalábbis  maguk is (speciális)
tulajdonságok.


\section{Telepítés}
Az OpenLDAP szerver és a kliensoldali programok egyaránt az \texttt{openldap} csomagban találhatóak, melyet az
\texttt{emerge openldap} paranccsal telepíthetünk Gentoo Linux esetén, azonban például Debian alatt már több csomagból
áll, ezek: \texttt{slapd} és \texttt{ldap-utils}, és a~telepítő egy~minimális fastruktúrát létrehoz a~\texttt{slapd.conf}
állománnyal együtt, ezeket azonban Gentoo-n nekünk kell megtennünk.

A beállítófájlok helye a \texttt{/etc/openldap} könyvtár Gentoo alatt és \texttt{/etc/ldap} Debian rendszer esetén.

\section{A szerver beállítófájlja: a slapd.conf}
A \texttt{/etc/openldap/slapd.conf} fájl tartalmazza a szerver beállításait, nagy vonalakban ezek a következők:

\begin{itemize}
\item Sémák (a használható attribútumokat adják meg)
\item Hozzáférés-vezérlési listák (\textsc{acl})
\item Titkosítás, hitelesítés és egyéb információk
\item A (háttér)adatbázis típusa, a keresést segítő index módja
\item Az \textsc{ldap} faszerkezet csúcsát jelölő név
\item A ``szuperfelhasználó'' (\texttt{rootdn}) egyedi neve és jelszava (mindent olvas, és ír, az \textsc{acl} nem érvényes rá)
\end{itemize}  

Figyeljünk arra, hogy ne olvashassa mindenki ezt a fájlt!

\paragraph{Séma} Az általunk használt séma az alábbi:

\begin{VerbExample}
include         /etc/openldap/schema/core.schema
include         /etc/openldap/schema/cosine.schema
include         /etc/openldap/schema/inetorgperson.schema
include         /etc/openldap/schema/nis.schema
include         /etc/openldap/schema/misc.schema
\end{VerbExample}
  
A felsoroltak közül az első a rendszernek kell lényegében, a következő három az általánosan használt attribútumokat
tartalmazza (többek között a felhasználók definiálásához). Az utolsó ``\texttt{misc}'', azaz nem megszokott, most az
ebben definiáltakkal lesz meghatározva az a cím, amelyre érkezett leveleket az adott felhasználó fogadhatja, illetve
melyre továbbküldheti (ha nem küldi tovább, akkor is szükséges). Az \textsc{ldap}-ban tárolt álnevek (aliases)
meghatározásához is szükséges.

\paragraph{Jogok (\textsc{acl})} A következő beállítás nagyon alapszintű, érdemes alaposabban utánaolvasni az OpenLDAP
leírásában, vagy épp az interneten, mert esetleg sokkal bonyolultabb hozzáférés-szabályozásra lesz szükség.

A jelszót csak saját maga tudja írni, valamint az admin.

\begin{VerbExample}
access to attrs=userPassword
    by dn="cn=admin,dc=panthernet" write
    by anonymous auth
    by self write
    by * none
\end{VerbExample}    

A következő két beállítás nagyjából azonos a jelenlegi esetben, csak az egyik szükséges!

\begin{VerbExample}
 access to dn.children="ou=People,dc=panthernet"
        by dn="cn=admin,dc=panthernet" write
        by * read

#access to *
#        by dn="cn=admin,dc=panthernet" write
#        by * read
\end{VerbExample}


Látszik, hogy az admin nevű felhasználónak joga van írni bármilyen bejegyzést. De ki is ő? A rendszerben van egy
kitüntetett felhasználó (\texttt{rootdn}, akinek a jogosultságát nem lehet szabályozni, bármit módosíthat, a
neve és a jelszava a\texttt{slapd.conf} fájlban adható meg. Éppen ezért csak bizonyos esetekben szabad használni, ha
máshogyan nem oldható meg a módosítás. Ebből következően szükség van egy kitüntetett felhasználóra is, amely az~\textsc{ldap}
adminisztrátor lesz, a Debian telepítésnek megfelelően \texttt{cn=admin,dc=domain,dc=hu} formában hozzuk létre.


\paragraph{Az adatbázis és a root ``user''} Az adatokat általában Berkeley adatbázis tartalmazza, mely nagyon
gyors, ámde nem hordozható, vagyis maguk az adatbázisfájlok nem vihetőek át másik gépre. Ehhez exportálni kell az
adatbázist, a másik gépen pedig importálni. A faszerkezet gyökerét, aminek a neve minden csúcs nevének végződése, a
\texttt{suffix} beállítás tartalmazza. A ``root'' felhasználót a ``\texttt{rootdn}'' és a ``\texttt{rootpw}'' pár
azonosítja.

\begin{VerbExample}
database        bdb
suffix          "dc=panthernet"
rootdn          "cn=Manager,dc=panthernet"
# Cleartext passwords, especially for the rootdn, should
# be avoid.  See slappasswd(8) and slapd.conf(5) for details.
# Use of strong authentication encouraged.
rootpw          {SSHA}blabla

checkpoint      32      30 # <kbyte> <min>


# The database directory MUST exist prior to running slapd AND
# should only be accessible by the slapd and slap tools.
# Mode 700 recommended.
directory       /var/lib/openldap-data
# Indices to maintain
index   objectClass     eq
#
LogLevel 0
\end{VerbExample}

Mint látható, a jelszót is meg kell adni. Lehetne titkosítás nélküli (cleartext) jelszót is megadni, de ez nem
célszerű, ugyanis ha esetleg valami hiba folytán a \texttt{slapd.conf} állományt más is olvashatja, akkor azonnal
megvan. Ha egy bonyolult jelszót választunk, és titkosítva tároljuk, akkor akár évekig, sőt esetleg évezredekig
tartana kitalálni a jelszót. Az \textsc{SSHA} egy megfelelő algoritmus, egyirányú, ezért
nem lehet a jelszót visszaállítani, csak bruteforce-szal törhető (vagy szótári támadással, ha rossz a jelszó). Tehát a
jó jelszó: kis-  és nagybetűk, számok vegyesen, lehetőleg egyéb nyomtatható karakterekkel (pl. \#, \&, ! stb.)
megtűzdelve, és legyen minél hosszabb. Mivel a jelszó hash formájában tárolódik, ezért több különböző jelszót megadva
azonos hash-t kapunk. \textsc{SHA1} esetén 160 bites a hash hossza, ennél hosszabb jelszavak esetén garantáltan lesz több
egyező is.


A jelszó előállítására szolgál a \texttt{slappaswd} parancs, ennek kimenetét kell bemásolni. Használata:\\
\texttt{slappasswd -h {SSHA}}


Az adatbázisról (\texttt{database} beállításról) még egyszer: az adatbázis lehet többféle, a legegyszerűbb az
\texttt{ldbm}, a legjobb választás azonban a  \texttt{bdb} (Berkeley DB), melynek fájljai a~\texttt{directory}
beállításnak megfelelő könyvtárban lesznek. A könyvtárra a \texttt{700} jogot ajánlott megadni, hogy ne tudja a
könyvtár tulajdonosán más is olvasni a fájlokat. Az  \texttt{index} a  kereséshez használt index.

A \texttt{LogLevel} mondja meg, mennyire bőbeszédűen (verbose) írja bele a syslog-ba, hogy mi történik. A ``0'' érték
hatására szinte semmit sem ír, hibakeresésnél jön jól.

\section{Az ldap.conf}
A szerverhez történő hozzáférés beállításait a \texttt{/etc/openldap/ldap.conf} állomány tartalmazza.
Ez a fájl mondja meg, hogy az \texttt{ldapsearch}, és hasonló parancsok alapesetben milyen géphez, milyen protokollal
és milyen végződésű helyhez (lásd a fentebbi \texttt{suffix}, itt \texttt{\textsc{base}}) csatlakozik. A fájl mindenki által
olvasható.

\begin{VerbExample}[frame=topline,label=ldap.conf]
BASE    dc=panthernet
URI     ldaps://ldaps.panthernet

TLS_CACERT /etc/openldap/ssl/cacert.pem

SASL_REALM panthernet
\end{VerbExample}

A \texttt{\textsc{base}} után a faszerkezet valamely csúcsának egyedi nevét kell megadni, amely alatti bejegyzéseken történik a
keresés (tehát részfát határoz meg). Az \texttt{URI} a szerver elérését adja meg protokollal együtt. A példában
a titkosított protokoll (ldaps) szerepel, az ``ldaps.panthernet'' nevű gépen érhető el a szerver, a titkosítás miatt meg
kell adni azt a~helyet, ahol a \textsc{ca} tanúsítványa található, ez a \texttt{\textsc{tls}\_\textsc{cacert}} beállítás.

A \texttt{\textsc{sasl}\_\textsc{realm}} \aref{cha:Kerberos}.\ fejezetben részletesen tárgyalt Kerberos beállításaihoz
tartozik, az alap OpenLDAP nem igényli, itt csupán a teljesség kedvéért szerepel.

\section{A rendszer felkészítése az \textsc{ldap} használatához}

A fentiek elvégzése után még csak a szerver beállítása és a kapcsolódás módja adott. Ahhoz, hogy az \textsc{ldap}-ban tárolt
felhasználók ténylegesen létezhessenek az adott gépen (pl. az \texttt{id} parancs lássa őket), fel kell telepíteni 
az ``nss\_ldap'' csomagot. Általában az \textsc{ldap}
szerver különálló gépen található, de nincs mindig így, ezért néha azon is be kell állítani a csomagot
(\filename{/etc/ldap.conf} tartalmazza a beállításait).

\subsection{nss-ldap}
A névfeloldáshoz és az \textsc{ldap}-ban tárolt jelszavak kezeléséhez nélkülözhetetlen az \texttt{nss-ldap} csomag. A
beállítófájlja a \filename{/etc/ldap.conf}, ennek néhány fontosabb beállítása:

\begin{VerbExample}[frame=topline,label=/etc/ldap.conf részlet]
base dc=panthernet
uri ldaps://ldaps.panthernet/
# The distinguished name to bind to the server with.
#Optional: default is to bind anonymously.
#binddn cn=proxyuser,dc=padl,dc=com
  
# The credentials to bind with.
# Optional: default is no credential.
#bindpw secret
\end{VerbExample}

Az első kettő lényeges minden esetben. Az egyik a szerver elérését mondja meg (\texttt{uri}): ldaps, azaz titkosított,
utána pedig a gép neve szerepel. A másik, a \texttt{base} pedig a faszerkezet egy csúcsát, ami alatt keres a rendszer
(\texttt{ou=People}, \texttt{ou=Group}). Alapesetben ez a teljes részfát jelenti, ám ez tovább módosítható lenne.

A \texttt{binddn} és a \texttt{bindpw} segítségével elérhető, hogy névtelenül ne lehessen a szerverhez csatlakozni,
onnan információkat gyűjteni, csak az itt megadott ``felhasználó'' és jelszó tudjon. Igazából a fa bármely csúcsa
lehet, ha annak van joga olvasni szinte bármit, kivéve például a \texttt{userPassword} tulajdonságot. Ehhez a sor
eleji ``\#'' karaktert (megjegyzés jele) ki kell törölni, és érvényes adatot írni a helyére.

A fájl többi részét nem szükséges módosítani.

\paragraph{nsswitch.conf} Az \textsc{ldap} szerverhez történő csatlakozást a fentebb látható módon garantáltuk, de ez
kevés. Még meg kell mondani, hogy az információkat onnan (is) vegye. Ezt a~\texttt{/etc/nsswitch.conf} fájl
módosításával tehető meg.

\begin{VerbExample}[frame=topline,label=/etc/nsswitch.conf \textsc{ldap}-ot használó része]
passwd:      files ldap
shadow:      files ldap
group:       files ldap
\end{VerbExample}

Vagyis a jelszavakat, shadow jelszavakat és a csoportokat egyrészt a helyi fájlokból, másrészt \textsc{ldap}-ból olvassa ki (az
elől álló nevek a \texttt{/etc/} alatti fájlneveket jelzik.

Ha nem akarjuk, hogy az \textsc{ldap}-ban tárolt jelszavakat használja a rendszer, akkor a~\texttt{passwd} beállításnál nem
szükséges a \texttt{ldap} megadása.

\section{Az adatok feltöltése}
Egyelőre még egy teljesen üres adatbázisunk van, viszont már minden megvan ahhoz, hogy a teljes rendszer működjön. Akár
le is kérdezhetjük az adatbázis tartalmát az \texttt{ldapsearch -x} parancs kiadásával. Természetesen a kimenet üres
lesz (pár megjegyzést nem számítva).

A mostani példában felhasználók, csoportok és email álnevek (álnevek) szerepelnek. Az ezek szülőcsúcsáig bezárólag
kézzel kell (érdemes) létrehozni a fát, utána már lehet szkriptekkel is beszúrni az adatokat.

\subsection{\textsc{ldap} bejegyzések hozzáadása}

Az \textsc{ldap} a bejegyzéseket \textsc{ldif} formátumban jeleníti meg (például az ldapsearch és \texttt{slapcat}
parancsok kimenete ilyen), és ezt a formátumot használva lehet adminisztrálni is. Ezért az összes bejegyzés (csúcs)
ilyen formában van itt is.

Ha elkészítettük a fájlt, amiben az \texttt{ldif} bejegyzés(ek) van(nak), akkor az \texttt{ldapadd} paranccsal hozzá lehet adni:\\
\texttt{ldapadd -xWD cn=Manager,dc=panthernet -f fájlnév.ldif}

Természetesen csak a legelső elemeket érdemes így létrehozni, utána pedig már másik felhasználót (a \texttt{-D} után egy
másik bejegyzés, csúcs nevét) érdemes megadni.


\subsection{Az alapvető bejegyzések}

Az alapvető bejegyzések a gyökértől kezdődően lefelé a csoportokat definiáló csúcsokig tartanak, valamint az admin és
Manager kitüntetett felhasználók (lehet, hogy a Manager felesleges is, ámde így biztosan működik).

Tekintsük először a gyökeret, s annak tulajdonságait.
A \texttt{dc} az egyedi név, az~\texttt{o} a~szervezet (organization) rövidítése.

\begin{VerbExample}
dn: dc=panthernet
objectClass: dcObject
objectClass: organization
dc: panthernet
o: PantherNetwork  
\end{VerbExample}


\noindent A következő \filename{slapd.conf} fájlban definiált ``felhasználó''. Az idézőjel oka, hogy valójában nem is
felhasználó, ezért is más az objectClass. Kettő is van neki, a második (\texttt{top}) csak annyit jelent, hogy az
objektumnak lehetnek gyerekei (bizonyos esetekben szükséges, bizonyos esetekben meg nem, ez erősen függ a többi megadott
objectClass tulajdonságtól). Ezért a \texttt{top} egy felesleges objectClass, de nem helytelen. Lehet, hogy enélkül is
menne a~rendszer...


\begin{VerbExample}
dn: cn=Manager,dc=panthernet
objectClass: organizationalRole
objectClass: top
cn: Manager
\end{VerbExample}

\noindent Most tekintsük az admin felhasználó bejegyzését, mely a Manager-től független adminisztrátori jogokkal
rendelkezik. Nem kötelező, mégis jobb, ha szét vannak választva a jogosultságok, akár több részre is. A
\texttt{simpleSecurityObject} tartalmazza a \texttt{userPassword} tulajdonságot, vagyis lehet lehet hitelesíteni
(jelszót ellenőrizni). A \texttt{description} mező pedig a leírás, melyet bármely objektum tartalmazhat, ámde használata
opcionális.


\begin{VerbExample}
dn: cn=admin,dc=panthernet
objectClass: organizationalRole
objectClass: simpleSecurityObject
cn: admin
description: LDAP administrator
userPassword: {SSHA}blabla
\end{VerbExample}


\noindent A rendszeren létező összes (valódi) felhasználó egyetlen bejegyzés alá kerül, ez pedig a~következő lesz, ahol
szervezeti egységet jelent az \textt{organizationalUnit}:

\begin{VerbExample}
dn: ou=People,dc=panthernet
objectClass: organizationalUnit
ou: People  
\end{VerbExample}


\noindent Az email aliasokat tartalmazó ág csúcsa:

\begin{VerbExample}
dn: ou=MailAliases,dc=panthernet
ou: MailAliases
objectClass: top
objectClass: organizationalUnit
\end{VerbExample}


\noindent A csoportokat tartalmazó szervezeti egység:

\begin{VerbExample}
dn: ou=Group,dc=panthernet
objectClass: organizationalUnit
ou: Group
\end{VerbExample}



\subsection{A többi bejegyzés}
Az előbb ismertetett bejegyzéseket létrehozva meg van minden, hogy felhasználókat létrehozzunk, sőt, bármilyen
információt tárolhatunk az \textsc{ldap} szerveren, bár most nem célunk. Eddig csak a Manager (azaz a \texttt{rootdn}) tudott
létrehozni csúcsokat, most már megtehetjük ezt az általunk létrehozott admin felhasználó segítségével.


Az első kettő egy-egy felhasználói bejegyzés (nagyjából így definiáltam a felhasználómat a notebookomon).
A példában nagyon sok tulajdonság kitöltetlen. Először tekintsük át, milyen osztályokba tartozik a felhasználó, és
melyek az egyes osztályok lényegesebb tulajdonságai.

Az \texttt{inetorgPerson} többek között a következő opcionális tulajdonságokat határozza meg
(\texttt{inetorgperson.schema}):
\texttt{homePhone}, \texttt{homePostalAddress}, \texttt{initials}, \texttt{mail},  \texttt{mobile},
\texttt{roomNumber}, \texttt{uid}.

A \textt{person} (\texttt{core.schema}) esetén az \texttt{sn}, \texttt{cn} attribútumokat kötelező definiálni
(vezetéknév az \texttt{sn}). Opcionális többek között: \texttt{ userPassword}, \texttt{telephoneNumber}.

Az \texttt{organizationalPerson} valamilyen szervezethez tartozó személyt jelent. Itt többek között
\texttt{telephoneNumber}, \texttt{ou} (azaz a szervezeti egység), \texttt{postalAddress} tulajdonságok szerepelnek.

A \texttt{inetLocalMailRecipient} a misc.schema-ban definiált, \texttt{mailLocalAddress}, \texttt{mailHost},
\texttt{mailRoutingAddress} az opcionális tulajdonságai. Ebből az első az, amire érkeznek a levelek, az utolsó meg
amire mennek. Ha helyi fiók (pl. panther), akkor a levelezőszervernek (pl. Cyrus) továbbítódik, különben a megadott
címre.

A \texttt{posixAccount} és \texttt{shadowAccount} tartalmazza a \texttt{/etc/passwd} és \texttt{/etc/shadow} fájlban
is megtalálható információkat

Ha a jelszavakat nem szeretnénk Kerberos-ban tárolni, ámde \textsc{ldap}-ban igen, akkor még egy osztályt meg kell adnunk, ez
pedig a \texttt{shadowAccount}. Most nincs szükségünk rá.

Mivel a nevem ékezeteket is tartalmaz, ezért UTF-8-ban megadtam, majd pedig Base64-gyel kódoltam az egyes mezők
értékét. Emiatt ezeknél a mezőknél nem egy, hanem két kettőspont szerepel.

\begin{VerbExample}
dn: uid=panther,ou=People,dc=panthernet
objectClass: top
objectClass: person
objectClass: inetOrgPerson
objectClass: organizationalPerson
objectClass: inetLocalMailRecipient
objectClass: posixAccount
cn:: VMOzdGggTMOhc3psw7MgQXR0aWxh
cn:: TMOhc3psw7MgQXR0aWxhIFTDs3Ro
uid: panther
mobile: 06
homePostalAddress: Erd
mailRoutingAddress: panther
mailLocalAddress: Toth.Laszlo.Attila
mailLocalAddress: Laszlo.Attila.Toth
gecos: Toth Laszlo Attila
sn:: VMOzdGg=
homeDirectory: /home/panther
loginShell: /bin/bash
uidNumber: 1000
gidNumber: 100
 givenName:: TMOhc3psw7MgQXR0aWxh
\end{VerbExample}

\noindent illetve

\begin{VerbExample}
dn: uid=parad,ou=People,dc=panthernet
objectClass: top
objectClass: person
objectClass: inetOrgPerson
objectClass: organizationalPerson
objectClass: inetLocalMailRecipient
objectClass: posixAccount
objectClass: shadowAccount
cn: Nagy Adrienn
cn: Adrienn Nagy
gecos: Nagy Adrienn
sn: Nagy
givenName: Adrienn
uid: parad
mailRoutingAddress: parad
mailLocalAddress: Nagy.Adrienn
mailLocalAddress: Adrienn.Nagy
homeDirectory: /home/parad
loginShell: /bin/bash
uidNumber: 1001
gidNumber: 100
\end{VerbExample}

\noindent Csoportokat is létrehozhatunk, ahol felsorolhatjuk a tagokat is uid szerint:

\begin{VerbExample}
dn: cn=testgrp,ou=Group,dc=panthernet
objectClass: top
objectClass: posixGroup
cn: testgrp
gidNumber: 10000
memberUid: panther
memberUid: parad
\end{VerbExample}


Végezetül lássunk egy példát a levelezésben használt álnevekre. A bejegyzés akár több álnevet is tartalmazhat
(\texttt{mailLocalAddress} tulajdonság), és akár több címre is továbbítódhat. Ez utóbbit a \texttt{mailRoutingAddress}
tulajdonsággal adhatjuk meg, amelyből az előzővel ellentétben egyetlen egy lehet, és az összes címzettet szóközzel
elválasztva meg kell adni utána.

\begin{VerbExample}
dn: uid=mindenki,ou=MailAliases,dc=panthernet
objectClass: account
objectClass: top
objectClass: inetLocalMailRecipient
mailLocalAddress: mindenki
mailLocalAddress: everyone
mailLocalAddress: everybody
uid: mindenki
mailRoutingAddress: panther parad
\end{VerbExample}


\subsection{Az \textsc{ldap} bejegyzések módosítása}

Sokszor menet közben kell felvenni újabb adatokat, vagy meglévőket módosítani, esetleg törölni néhányat. Ezt mind a
következő paranccsal tehetjük meg:\\
\texttt{ldapmodify -xWD cn=Manager,dc=panthernet -f fájlnév.ldif}

Nézzük sorra, hogy mit hogyan lehet:

\noindent  --- Attribútum hozzáadása:

\begin{VerbExample}
dn: uid=panther,ou=People,dc=panthernet
changeType: modify
add: userPassword
userPassword: {KERBEROS}panther@PANTHERNET
\end{VerbExample}

\noindent --- Attribútum cseréje:
\begin{VerbExample}
dn: uid=panther,ou=People,dc=panthernet
changeType: modify
replace: userPassword
userPassword: {KERBEROS}panther2@PANTHERNET
\end{VerbExample}

\noindent --- Lehet törölni is egy attribútumot, például:

\begin{VerbExample}
dn: uid=panther,ou=People,dc=panthernet
changeType: modify
delete: userPassword
\end{VerbExample}


\noindent --- Ha egy adott attribútumot (amiből több különböző értékű is lehet) kellene, akkor:

\begin{VerbExample}
dn: uid=panther,ou=People,dc=panthernet
changeType: modify
delete: mail
mail: panther@elte.hu
\end{VerbExample}

\noindent --- Több is egyszerre:

\begin{VerbExample}
dn: uid=panther,ou=People,dc=panthernet
changeType: modify
add: userPassword
userPassword: {KERBEROS}panther@PANTHERNET
-
replace: homePostalAddress
homePostalAddress: Erd, stb.
\end{VerbExample}

\noindent --- A teljes bejegyzés törlése:
  
\begin{VerbExample}
dn: uid=panther,ou=People,dc=panthernet
changeType: delete
\end{VerbExample}



\section{Titkosítás}

Az eddigiek alapján nagyjából működőképes a rendszerünk, azonban vannak
hiányosságai. Az egyik az, hogy a szerver és a kliens között titkosítatlan a kommunikáció, ezért egy harmadik személy
is hozzáférhet az adatokhoz, ha valahogyan meg tudja szerezni a kliens-szerver kapcsolat csomagjait. A szerver a \textsc{tls}
(és \textsc{ssl}) titkosítást támogatja (Transport Layer Security és Secure Socket Layer kifejezések rövidítései).

A \textsc{tls} beállításhoz szükség van tanúsítványokra (ezek kezelését később tárgyaljuk). A~beállítások a~következők:

\begin{VerbExample}
TLSCACertificateFile /etc/openldap/ssl/cacert.pem
TLSCertificateFile /etc/openldap/ssl/ldaps.panthernet.crt
TLSCertificateKeyFile /etc/openldap/ssl/ldaps.panthernet.key
#TLSCipherSuite HIGH,MEDIUM
\end{VerbExample}  

Az első sor jelzi a tanúsítvány-kiszolgáló (CA) tanúsítványát (certificate). A második az OpenLDAP-ét, a harmadik pedig
az OpenLDAP titkos kulcsa (key). Ennek a fájlnak kötelezően 600 jogúnak kell lennie, nem szabad, hogy bárki is olvasni
tudja. Az utolsó sor csak a~közepes és erős titkosító algoritmusokat engedélyezi (ha nincs ott a kettős kereszt (\#)).


%\section{Replikáció}

%TODO



\section{Kerberos alapú hitelesítés}

A Kerberos előnye, hogy egyszer kell a felhasználót hitelesíteni, utána hozzáférhet más szolgáltatásokhoz, anélkül, hogy
megadná újból és újból a jelszavát. Részletesen lásd \aref{cha:Kerberos}. fejezetben. Az \textsc{ldap} szerverhez történő
hozzáférés is szabályozható a következőkben (valamint egy további opció is szerepel a példában):

\begin{VerbExample}
sasl-realm panthernet
sasl-regexp uid=([^,]+),.*cn=GSSAPI,.* uid=$1,ou=People,dc=panthernet
sasl-host zeratul.panthernet
\end{VerbExample}

A \texttt{sasl-realm} azt az ún. kerberos realm-ot határozza meg, amiben a kerberos bejegyzések vannak. A
\texttt{sasl-host} a kiszolgáló helyét mondja meg, \texttt{sasl-regexp} pedig azt, hogy hogyan lehet megtalálni azt a
bejegyzést, aki csatlakozott a szerverhez.

% Local Variables:
% fill-column: 120
% TeX-master: t
% End:

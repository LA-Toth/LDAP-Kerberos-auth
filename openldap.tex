\section{Az OpenLDAP}

Az itt ismertetett rendszer egyik központi része az OpenLDAP szerver, amely az LDAP protokollt ({\em Lightweight
  Directory Access Protocol}) valósítja meg. Ez egy ``könyvtárszolgáltatás''
(directory service) elérésére szolgál, amelynek jellemzője, hogy olvasásra, keresésre, böngészésre (átnézésre)
optimalizált adatbázist használ az adatok tárolására. Az adatbázis lényegében \emph{bármilyen adatot} tartalmhazhat
\emph{bármiről}, az egyetlen megkötés, hogy faszerkezetű hierarchiába kell szervezni.

Az OpenLDAP szerverhez általában TCP/IP kapcsolaton keresztül lehet csatlakozni, azonban a helyi gépen (amelyen az
ldap kiszolgáló található), unix socketen keresztül is történhet a kapcsolódás.

\subsection{Mi LDAP?}

Az LDAP egy hierarchikusan (faszerkezetbe) szervezett adatbázis elérésére szolgáló protokoll. A fában tárolt adatokat
egy-egy egyedi név: {\em Distinguished Name (DN)} azonosítja, amely név több részből áll, a  különböző részek alapján
történik a bejegyzések (objektumok) fába szervezése. Minden bejegyzéshez tartozik egy vagy több
tulajdonság, amit egy {\em kulcs-érték} pár határoz meg. A kulcs lehet például a ``common name'' (cn) - mindennapi
név, a hozzá tartozó érték meg ``Gipsz Jakab'', vagy épp a levélcím megadására szolgáló ``mail'' mező. Egy-egy
tulajdonság (lényegében maga a kulcs) többször is szerepelhet egy objektumon belül.

A fa csúcsainak elhagyható és kötelező tulajdonságait az objektumok (csúcsok) osztályai írják le
(tulajdonságosztályok). A nevük: \texttt{objectclass}, és a jelölés erejéig legalábbis  maguk is (speciális)
tulajdonságok.


\subsection{Telepítés, beállítás}
Az OpenLDAP szerver és a kliensoldali programok egyaránt az \texttt{openldap} csomagban találhatóak, melyet az
\texttt{emerge openldap} paranccsal telepíthetünk Gentoo linux esetén, azonban például Debian alatt már több csomagból
áll, ezek: \texttt{slapd} és \texttt{ldap-utils}, egy minimális fastruktúrát létrehoz a \texttt{slapd.conf}
állománnyal együtt, amit a Gentoo nem tesz meg.

A beállítófájlok helye a \texttt{/etc/openldap} könyvtár Gentoo alatt, Debianban ez a \texttt{/etc/ldap}.

\subsubsection{A /etc/openldap/slapd.conf}
A \texttt{slapd.conf} fájl tartalmazza a szerver beállításait, nagy vonalakban ezek a következők:

\begin{itemize}
\item Sémák (a használható attribútumokat adják meg)
\item Hozzáférés-vezérlési listák (ACL)
\item Titkosítás, hitelesítés és egyéb információk
\item A (háttér)adatbázis típusa, a keresést segítő index módja
\item Az LDAP faszerkezet csúcsát jelölő név
\item A ``szuperfelhasználó'' (\texttt{rootdn}) egyedi neve és jelszava (mindent olvas, és ír, az ACL nem érvényes rá)
\end{itemize}  

Figyeljünk arra, hogy ne olvashassa minenki ezt a fájlt!

\paragraph{Séma} Az általunk használt séma az alábbi:

\begin{Verbatim}
include         /etc/openldap/schema/core.schema
include         /etc/openldap/schema/cosine.schema
include         /etc/openldap/schema/inetorgperson.schema
include         /etc/openldap/schema/nis.schema
include         /etc/openldap/schema/misc.schema
\end{Verbatim}
  
A felsoroltak közül az első a rendszernek kell lényegében, a következő három az általánosan használt attribútumokat tartalmazza
(többek között a felhasználók definiálásához). Az utolsó ``\texttt{misc}'', azaz nem megszokott, most az ebben
definiáltakkal lesz meghatározva az a cím, amelyre érkezett leveleket az adott felhasználó fogadhatja, illetve melyre
továbbküldheti (ha nem küldi tovább, akkor is szükséges). Az ldap-ban tárolt álnevek (aliasok) meghatározásához is
szükséges.

\paragraph{Jogok (ACL)} A következő beállítás nagyon alapszintű, érdemes alaposabban utánaolvasni az openldap
leírásában, vagy épp az interneten, mert esetleg sokkal bonyolultabb hozzáférés-szabályozásra lesz szükség.

A jelszót csak saját maga tudja írni, valamint az admin.

\begin{Verbatim}
access to attrs=userPassword
    by dn="cn=admin,dc=panthernet" write
    by anonymous auth
    by self write
    by * none
\end{Verbatim}    

A következő két beállítás nagyjából azonos a jelenlegi esetben, csak az egyik szükséges!

\begin{Verbatim}
access to dn.children="ou=People,dc=panthernet"
        by dn="cn=admin,dc=panthernet" write
        by * read

#access to *
#        by dn="cn=admin,dc=panthernet" write
#        by * read
\end{Verbatim}


Látszik, hogy az admin nevű felhasználónak joga van írni bármilyen bejegyzést. De ki is ő? A rendszerben van egy
kitüntetett felhasználó (\texttt{rootdn}, akinek a jogosultságát nem lehet szabályozni, bármit módosíthat, a
neve és a jelszava a\texttt{slapd.conf} fájlban adható meg. Éppen ezért csak bizonyos esetekben szabad használni, ha
máshogyan nem oldható meg a módosítás. Ebből következően szükség van egy kitüntetett felhasználóra is, amely az LDAP
adminisztrátor lesz, a Debian telepítésnek megfelelően \texttt{cn=admin,dc=domain,dc=hu} formában hozzuk létre.


\paragraph{Az adatbázis és a root ``user''} Az adatokat általtalában Berkeley adatbázis tartalmazza, mely nagyon
gyors, ámde nem hordozható, vagyis maguk az adatbázisfájlok nem vihetőek át másik gépre. Ehhez exportálni kell az
adatbázist, a másik gépen pedig importálni. A faszerkezet gyökerét, aminek a neve minden csúcs nevének végződése, a
\texttt{suffix} beállítás tartalmazza. A ``root'' felhasználót a ``\texttt{rootdn}'' és a ``\texttt{rootpw}'' pár
azonosítja.

\begin{Verbatim}
database        bdb
suffix          "dc=panthernet"
rootdn          "cn=Manager,dc=panthernet"
# Cleartext passwords, especially for the rootdn, should
# be avoid.  See slappasswd(8) and slapd.conf(5) for details.
# Use of strong authentication encouraged.
rootpw          {SSHA}blabla

checkpoint      32      30 # <kbyte> <min>


# The database directory MUST exist prior to running slapd AND
# should only be accessible by the slapd and slap tools.
# Mode 700 recommended.
directory       /var/lib/openldap-data
# Indices to maintain
index   objectClass     eq
\end{Verbatim}

Mint látható, a jelszót is meg kell adni. Lehetne titkosítás nélküli (cleartext) jelszót is megadni, de ez nem
célszerű, ugyanis ha esetleg valami hiba folytán a \texttt{slapd.conf} állományt más is olvashatja, akkor azonnal
megvan. Ha egy bonyolult helszót választunk, és titkosítva tároljuk, akkor akár évekig, sőt esetleg évezredekig
tartana kitalálni a jelszót. Az SSHA egy megfelelő algoritmus, egyirányú, ezért
nem lehet a jelszót visszaállítani, csak bruteforce-szal törhető (vagy szótári támadással, ha rossz a jelszó). Tehát a
jó jelszó: kis-  és nagybetűk, számok vegyesen, lehetőleg egyéb nyomtatható karakterekkel (pl \#, \&, ! stb.)
megtűzdelve, és legyen minél hosszabb. Mivel a jelszó hash formájában tárolódik, ezért több különböző jelszót megadva
azonos hash-t kapunk. SHA1 esetén 160 bites a hash hossza, ennél hosszabb jelszavak esetén garantáltan lesz több
egyező is.


A jelszó előállítására szolgál a \texttt{slappaswd} parancs, ennek kimenetét kell bemásolni. Használata:\\
\ut{slappasswd -h {SSHA}}


Az adatbázisról (\texttt{database} beállításról) mégegyzser: az adatbázis lehet többféle, a legegyszerűbb az
\texttt{ldbm}, a legjobb választás azonban a  \texttt{bdb} (Berkeley DB), melynek fájljai a \texttt{directory}
beállításnak megfelelelő könyvtárban lesznek. A könyvtárra a \texttt{700} jogot ajánlott megadni, hogy ne tudja a
könyvtár tulajdonosán más is olvasni a fájlokat. Az  \texttt{index} a  kereséshez használt index.

\paragraph{Kerberos, TLS, és ami még kimaradt} Az eddigiek alapján nagyjából működőképes a rendszerünk, azonban vannak
hiányosságai. Az egyik az, hogy a szerver és a kliens között titkosítatlan a kommunikáció, ezért egy harmadik személy
is hozzáférhet az adatokhoz, ha valahogyan meg tudja szerezni a kliens-szerver kapcsolat csomagjait. A szerver a TLS
(és SSL) titkosítást támogatja (Transport Layer Security és Secure Socket Layer kifejezések rövidítései).

A TLS beállításhoz szükség van tanúsítványokra (ezek kezelését később tárgyaljuk). A beállítások a következők:

\begin{Verbatim}
TLSCACertificateFile /etc/openldap/ssl/cacert.pem
TLSCertificateFile /etc/openldap/ssl/ldaps.panthernet.crt
TLSCertificateKeyFile /etc/openldap/ssl/ldaps.panthernet.key
#TLSCipherSuite HIGH,MEDIUM
\end{Verbatim}  

Az első sor jelzi a tanúsítvány-kiszolgáló (CA) tanúsítványát (certificate). A második az openldapét, a harmadik pedig
az openldap titkos kulcsa (key). Ennek a fájlnak kötelezően 600 jogúnak kell lennie, nem szabad, hogy bárki is olvasni
tudja. Az utolsó sor csak a közepes és erős titkosító algoritmusokat engedélyezi (ha nincs ott a kettőskereszt (\#)).


Ha kerberossal hitelesített a felhasználó, aki hozzáfér a jelszóhoz és módosítja, akkor egy-egy rondább jelszót elég
csak bejelentkezésenként egyszer megadni, minden egyéb szolgáltatásnál már nem szükséges.


A maradék beállítás:

\ut{sasl-realm panthernet\\
  sasl-regexp uid=([\^,]+),.*cn=GSSAPI uid=\$1,ou=People,dc=panthernet\\
  sasl-host zeratul.panthernet\\
  LogLevel 0}

A \texttt{LogLevel} mondja meg, mennyire bőbeszédűen (verbose) írja bele a syslogba, hogy mi történik. A ``0'' érték
hatására szinte semmit sem ír, hibakeresésnél jön jól.

A \texttt{sasl-realm} azt az ún. realm-ot határozza meg, amiben a kerberos bejegyzések vannak. A \texttt{sasl-host} a
kiszolgáló helyét mondja meg, \texttt{sasl-regexp} pedig azt, hogy hogyan lehet megtalálni azt a bejegyzést, aki
csatlakozott a szerverhez.


\subsubsection{A /etc/openldap/ldap.conf}
Ez a fájl mondja meg, hogy az \texttt{ldapsearch}, és hasonló parancsok alapesetben milyen géphez, milyen protokollal
és milyen végződésű helyhez (lásd a fentebbi \texttt{suffix}, itt \texttt{BASE}) csatlakozik. A fájl mindenki által
olvasható.

\ut{BASE    dc=panthernet\\
  URI     ldaps://ldaps.panthernet\\
  TLS\_CACERT /etc/openldap/ssl/cacert.pem }

A \texttt{BASE} az, ami alatt keres (részfát határoz meg), a \texttt{URI} azt, hogy titkosított protokollon (ldaps), az
``ldaps.panthernet'' nevű gépen érhető el a szerver. A titkosítás miatt meg kell adni azt a helyet, ahol a CA
tanúsítványa található, ez a \texttt{TLS\_CACERT} beállítás.

\subsection{A rendszer felkészítése az ldap használatához}
A fentiek elvégzése után még csak a szerver beállítása és a kapcsolódás módja adott. Ahhoz, hogy az ldapban tárolt
felhasználók ténylegesen létezhessenek az adott gépen, fel kell rakni az ``nss\_ldap'' csomagot. Általában az ldap
szerver különálló gépen található, de nincs mindig így, ezért néha azon is be kell állítani a csomagot
(\filename{/etc/ldap.conf} tartalmazza a beállításait).

\subsubsection{nss-ldap}
A \filename{/etc/ldap.conf} néhány fontos beállítása:

\ut{base dc=panthernet\\
  uri ldaps://ldaps.panthernet/\\
  \# The distinguished name to bind to the server with.\\
  \# Optional: default is to bind anonymously.\\
  \#binddn cn=proxyuser,dc=padl,dc=com\\
  \\
  \# The credentials to bind with.\\
  \# Optional: default is no credential.\\
  \#bindpw secret}

Az első kettő lényeges. Ez egyik szerver elérését mondja meg (\texttt{uri}): ldaps, azaz titkosított, utána a gép neve,
a másik, a \texttt{base} pedig a faszerkezet egy csúcsát, ami alatt keres a rendszer (\texttt{ou=People},
\texttt{ou=Group}).

A \texttt{binddn} és a \texttt{bindpw} segítségével elérhető, hogy névtelenül ne lehessen a szerverhez csatlakozni,
onnan információkat gyűjteni, csak az itt megadott ``felhasználó'' és jelszó tudjon. Igazából a fa bármely csúcsa
lehet, ha annak van joga olvasni szinte bármit, kivéve például a \texttt{userPassword} tulajdonságot. Ehhez a sor
eleji ``\#'' karaktert (megjegyzés jele) ki kell törölni, és érvényes adatot írni a helyére.

A fájl többi részét nem kell (de lehet) módosítani.


\paragraph{nsswitch.conf} Az ldap szerverhez történő csatlakozást a fentebb látható módon garantáltuk, de ez
kevés. Még meg kell mondani, hogy az információkat onnan (is) vegye. Ezt a \filename{/etc/nsswitch.conf} fájl
módosításával tehető meg.

\ut{\begin{tabular}{ll}passwd:& files ldap\\
    shadow:& files ldap\\
    group: & files ldap\end{tabular}}

Vagyis a jelszavakat, shadow jelszavakat és a csoportokat egyrészt a helyi fájlokból, másrészt ldapból olvassa ki (az
elől álló nevek a \texttt{/etc/} alatti fájlneveket jelzik.

\subsection{Az adatok feltöltése}
Ha ideiglenesen a TLS-re és a Kerberos (SASL) realm-ra vonatkozó információkat kivesszük, akkor már most tesztelhető a
rendszer a következő utasítással (keresés), ami ``üres'' kimenetet ad.\\
\texttt{ldapsearch -x}

A mostani példában felhasználók, csoportok és email aliasok (álnevek) szerepelnek. Az ezek szülőcsúcsáig bezárólag
kézzel kell (érdemes) létrehozni a fát, utána már lehet szkriptekkel is beszúrni az adatokat.



\subsubsection{Az LDAP bejegyzések hozzáadása}

Az ldap a bejegyzéseket LDIF formátumban jeleníti meg (például az ldapsearch és slapcat parancsok kimenete ilyen), és
ezt a formátumot használva lehet adminisztrálni is. Ezért az összes bejegyzés (csúcs) ilyen formában van itt is.

A \texttt{dc} az egyedi név, az \texttt{o} a szervezet (organization) rövidítése. A fenti objektum (csúcs) a fa
gyökere:

\ut{dn: dc=panthernet\\
  objectClass: dcObject\\
  objectClass: organization\\
  dc: panthernet\\
  o: PantherNetwork}


Ez a \filename{slapd.conf} fájlban definiált ``felhasználó''. Valójában nem is felhasználó, ezért is más az
objectClass. Kettő is van neki, a második (\texttt{top}) csak annyit jelent, hogy az objektumnak lehetnek
gyerekei. Ezért ez egy felesleges objectClass, de nem helytelen.

\ut{dn: cn=Manager,dc=panthernet\\
  objectClass: organizationalRole\\
  objectClass: top\\
  cn: Manager}

A Manager-től független adminisztrátori jogokkal rendelkező bejegyzés. Nem kötelező, mégis jobb, ha szét vannak
választva a jogosultságok, akár több részre is.\\ A \texttt{simpleSecurityObject} tartalmazza a \texttt{userPassword}
tulajdonságot. A \texttt{description} mező pedig a leírás, melyet bármely objektum tartalmazhat.

\ut{dn: cn=admin,dc=panthernet\\
  objectClass: organizationalRole\\
  objectClass: simpleSecurityObject\\
  cn: admin\\
  description: LDAP administrator\\
  userPassword: {SSHA}blabla}


Ez a bejegyzés alá kerül az összes (valódi) felhasználó. Szervezeti egységet jelent az \textt{organizationalUnit}:

\ut{dn: ou=People,dc=panthernet\\
  objectClass: organizationalUnit\\
  ou: People}

A következő egy felhasználói bejegyzés. Nagyon sok tulajdonság kitöltetlen. A \texttt{top} jelentése már elhangzott.

Az \texttt{inetorgPerson} többek között a következő opcionális tulajdonságokat határozza meg (inetorgperson.schema):
\texttt{homePhone}, \texttt{homePostalAddress}, \texttt{initials}, \texttt{mail},  \texttt{mobile},
\texttt{roomNumber}, \texttt{uid}.

A \textt{person} (\texttt{core.schema}) esetén az \texttt{sn}, \texttt{cn} attribútumokat kötelező definiálni
(vezetéknév az sn). Opcionális többek között: \texttt{ userPassword}, \texttt{telephoneNumber}.


Az \texttt{organizationalPerson} valamilyen szervezethez tartozó személyt jelent. Itt többek között
\texttt{telephoneNumber}, \texttt{ou} (azaz a szervezeti egység), \texttt{postalAddress} tulajdonságok szerepelnek.

A \texttt{inetLocalMailRecipient} a misc.schema-ban definiált, \texttt{mailLocalAddress}, \texttt{mailHost} ,
\texttt{mailRoutingAddress} az opcionális tulajdonságai. Ebből az első az, amire érkeznek a levelek, az utolsó meg
amire mennek. Ha helyi fiók (pl. panther), akkor a levelezőszervernek (pl. Cyrus) továbbítódik, különben a megadott
címre.

A \texttt{posixAccount} és \texttt{shadowAccount} tartalmazza a \texttt{/etc/passwd} és \texttt{/etc/shadow} fájlban
is megtalálható információjat.


\ut{dn: uid=panther,ou=People,dc=panthernet\\
  objectClass: top\\
  objectClass: person\\
  objectClass: inetOrgPerson\\
  objectClass: organizationalPerson\\
  objectClass: inetLocalMailRecipient\\
  objectClass: posixAccount\\
  objectClass: shadowAccount\\
  cn:: VMOzdGggTMOhc3psw7MgQXR0aWxh\\
  cn:: TMOhc3psw7MgQXR0aWxhIFTDs3Ro\\
  uid: panther\\
  mobile: 06\\
  homePostalAddress: Erd\\
  mailRoutingAddress: panther\\
  mailLocalAddress: Toth.Laszlo.Attila\\
  mailLocalAddress: Laszlo.Attila.Toth\\
  gecos: Toth Laszlo Attila\\
  sn:: VMOzdGg=\\
  homeDirectory: /home/panther\\
  loginShell: /bin/bash\\
  uidNumber: 1000\\
  gidNumber: 100\\
  givenName:: TMOhc3psw7MgQXR0aWxh}

Illetve:

\ut{dn: uid=parad,ou=People,dc=panthernet\\
  objectClass: top\\
  objectClass: person\\
  objectClass: inetOrgPerson\\
  objectClass: organizationalPerson\\
  objectClass: inetLocalMailRecipient\\
  objectClass: posixAccount\\
  objectClass: shadowAccount\\
  cn: Nagy Adrienn\\
  cn: Adrienn Nagy\\
  gecos: Nagy Adrienn\\
  sn: Nagy\\
  givenName: Adrienn\\
  uid: parad\\
  mailRoutingAddress: parad\\
  mailLocalAddress: Nagy.Adrienn\\
  mailLocalAddress: Adrienn.Nagy\\
  homeDirectory: /home/parad\\
  loginShell: /bin/bash\\
  uidNumber: 1001\\
  gidNumber: 100}

\newpage
Az email aliasokat tartalmazó ág csúcsa:

\ut{ dn: ou=MailAliases,dc=panthernet\\
  ou: MailAliases\\
  objectClass: top\\
  objectClass: organizationalUnit
}

Egy alias az account objectClass az uid attribútum miatt kell. A mailRoutingAddress-ben szóközzel elválasztott címekre
továbbítódik a levél:

\ut{dn: uid=mindenki,ou=MailAliases,dc=panthernet\\
  objectClass: account\\
  objectClass: top\\
  objectClass: inetLocalMailRecipient\\
  mailLocalAddress: mindenki\\
  mailLocalAddress: everyone\\
  mailLocalAddress: everybody\\
  uid: mindenki\\
  mailRoutingAddress: panther}

A csoportokat tartalmazó szervezeti egység:

\ut{dn: ou=Group,dc=panthernet\\
  objectClass: organizationalUnit\\
  ou: Group}

Egy csoportbejegyzés (több felhasználó is a tagja):

\ut{dn: cn=testgrp,ou=Group,dc=panthernet\\
  objectClass: top\\
  objectClass: posixGroup\\
  cn: testgrp\\
  gidNumber: 10000\\
  memberUid: panther\\
  memberUid: parad}


Ezeket a bejegyzéseket fájlba mentve egyből hozzáadhatjuk a futó ldap szerverhez kapcsolódva:\\
\texttt{ldapadd -xWD cn=Manager,dc=panthernet -f fájlnév.ldif}



\subsubsection{Az LDAP bejegyzések módosítása}

A módosításokat is LDIF formában kell megadni, a használt parancs:
\texttt{ldapmodify -xWD cn=Manager,dc=panthernet -f fájlnév.ldif}

Attribútum hozzáadása:

\ut{dn: uid=panther,ou=People,dc=panthernet\\
  changeType: modify\\
  add: userPassword\\
  userPassword: \{KERBEROS\}panther@PANTHERNET\}}

\newpage
Lehet törölni is egy attribútumot, például:

\ut{dn: uid=panther,ou=People,dc=panthernet\\
  changeType: modify\\
  delete: userPassword}

Ha egy adott attribútumot (amiből több különböző értékű is lehet) kellene, akkor:

\ut{dn: uid=panther,ou=People,dc=panthernet\\
  changeType: modify\\
  delete: mail\\
  mail: panther@elte.hu}

A teljes bejegyzés törlése:

\ut{dn: uid=panther,ou=People,dc=panthernet\\
  changeType: delete}

% Local Variables:
% fill-column: 120
% TeX-master: t
% End:

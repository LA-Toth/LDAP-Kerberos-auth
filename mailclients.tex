\chapter{Levelezőkliensek beállítása}
\section{Mutt}
A mutt rendszerszintű beállítófájljában, a \texttt{/etc/mutt/Muttrc}-ben a következőket állítjuk be:

\ut{set certificate\_file="/etc/openldap/ssl/cacert.pem"\\
  set move=no
  set hostname=panthernet\\
  set ssl\_starttls=yes\\
  set folder=\$MAIL
}

Az első a CA tanúsítványát jelöli, a 3. azért kell, hogy ne a gépnév.domain alakú email cím legyen, ha elküld a
rendszer egy levelet, a 4. az SSL/TLS titkosítást állítja be, végül az utolsó az IMAP könyvtár helye (INBOX), ami egy
környezeti változó:

\ut{export MAIL=\{\$USER@zeratul.panthernet\}INBOX}

\section{Thunderbird}
A Thunderbird címjegyzékében megadható LDAP kiszolgáló is, ilyenkor kereshetőek a szerveren található bejegyzések.

\subsection{A kiszolgáló megadása}
A \emph{kiszolgáló tulajdonságai} ablakban megadható a kapcsolat tulajdonságai, a következők szerint:

\paragraph{Általános fül} Itt a kapcsolat minimális tulajdonságai állíthatóak be.
\begin{Verbatim}
Név: Panthernet
Gépnév: ldaps.panthernet
Alap DN: ou=People,dc=Panthernet
Port száma: 389
\end{Verbatim}

Ennyi elég, a gépnév (Host) a kiszolgáló neve vagy IP címe, az alap DN (Base DN) az a csúcs, amely alatt keres a
Thunderbird. A port száma pedig ahol a kiszolgáló megtalálható.

Előfordulhat, hogy névtelen kapcsolódás tiltott, ekkor meg kell adni a \emph{Bejelentkezési DN} beállítást is,
pl. \texttt{uid=panther,ou=People,dc=panthernet}. Az első csatlakozáskor kéri a jelszót is. Végül megadható az is, hogy
SSL titkosítást használjon-e. Ebben az setben a jelszavak, adatok nem mindenki által olvashatóan, hanem kódoltan mennek a hálózaton, ezért érdemes használni.

\paragraph{Haladó fül} Néhány további beállítás adható meg, egyrészt a \emph{Hatókör (Scope)}, amely lehet \emph{Egy
  szint} (One level?), illetve \emph{Részfa} (Subtree), másrészt a keresés szűkítésére szolgáló szűrő. Alapesetben ez
  \emph{objectClass=*)}, azonban szűkíthető, hogy csak olyan bejegyzéseket listázzon, amelyek ténylegesen egy
  kapcsolatot (vagy bejelentkezési azonosítót) jelölnek: \emph{(objectClass=inetOrgPerson)}.

% Local Variables:
% fill-column: 120
% TeX-master: t
% End:

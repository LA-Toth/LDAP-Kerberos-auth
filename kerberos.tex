\chapter{Kerberos}

Egy részletes leírás itt olvasható: http://cfhay.inf.elte.hu/kerberos/leiras.html

A Kerberos a felhasználók (és szolgáltatások) biztonságos hitelesítésére szolgál. Amikor valaki bejelentkezik a gépre
a jelszójával, akkor kap egy őt azonosító \emph{ticket}-et, jegyet. Ezzel igényelhet továbbiakat, például amikor
továbbssh-zik egy másikra, vagy éppen elolvassa a levleleit.

A DNS-hez hasonlóan itt is tartományok vannak, csak ezeket \emph{realm}-oknak hívják. Például az example.com-hoz
tartozó realm EXAMPLE.COM (vagyis a domain név nagybetűs alakja, megszokásból, ugyanis nem kötelező így
eljárni). Nálam ez PANTHERNET.

Bejelentkezés után kapott ``jegy'' a \texttt{krbtgt/PANTHERNET@PANTHERNET} lesz. Itt a név a krbtgt (pontosabban
krbtgt/PANTHERNET) és a realm a ``@'' után álló rész, ami szintén PANTHERNET. Az adott felhasználó,
pl. \texttt{panther} bejegyzése a \texttt{panther@PANTHERNET} alakú, a néven kívül további tulajdonságok is tartoznak:
mikor jár le a jelszó, milyen lehet (bármilyen, vagy pl. kell kisbetű is, szám is, stb.). Ez a kettő szorosan
összekapcsolódik. Az utóbbi a felhasználót azonosítja, az előbbi a kerberizált szolgáltatások elérésére szolgál, ezért
is a neve \emph{ticket granting ticket}, vagyis további jegyeket biztosító jegy.

Egy realm nem csak felhasználókat tartalmaz, hanem szolgáltatásokat is, ezek formája:
\texttt{szolgáltatásnév/gépnév@REALM}. Ilyen például a \texttt{imaps/zeratul.panthernet@PANTHERNET}. Ezt a jegyet
akkor kapjuk meg, ha már egyszer bejelentkeztünk az imap szerverre. SSH szerver használata esetén pédául a
\texttt{host/fenix.panthernet@PANTHERNET} jegyet kapjuk meg.

Ebben a dokumentumban a \emph{mit-krb5} változatról lesz szó, viszont létezik másik implementáció is, a \emph{heimdal}.

A jegyet nevezhetjük kulcsnak is, és akkor kulcskiosztó szerverünk van (KDC).
\section{A kerberos beállítása}
A legfontosabb beállásokat a \texttt{/etc/krb5.conf} tartalmazza.

\subsection{/etc/krb5.conf illetve DNS bejegyzések}
A fájlt mindenki olvashatja. Nálam ezek a beállítások vannak:

\ut{[libdefaults]\\
  \hspace*{2cm}ticket\_lifetime = 86400\\
  \hspace*{2cm}default\_realm = PANTHERNET\\
  \hspace*{2cm}forwardable = true\newline\newline
	  [realms]\\
	  \hspace*{2cm}PANTHERNET = \{\\
	  \hspace*{2cm}kdc = 10.0.0.2:88\\
	  \hspace*{2cm}admin\_server = 10.0.0.2:749\\
	  \hspace*{2cm}\}\newline\newline
		  [domain\_realm]\\
		  \hspace*{2cm}.panthernet = PANTHERNET\\
		  \hspace*{2cm}panthernet = PANTHERNET\\
		  \hspace*{2cm}localhost = PANTHERNET\newline\newline
			  [logging]\\
			  \hspace*{2cm}kdc = FILE:/var/log/krb5kdc.log\\
			  \hspace*{2cm}admin\_server = FILE:/var/log/kadmin.log\\
			  \hspace*{2cm}default = FILE:/var/log/krb5lib.log}

A logolás (logging) nem túl érdekes. A libdefaults tartalmazza az alapértelmezett beállításokat. Az első sor a jegy
élettartamát adja meg, mely most egy nap. A következő az alapértelmezett realm, amely azért szükséges, hogy ne kelljen
mindig kiírni, melyik realmról van szó. Az utolsó pedig azt jelzi, hogy a kapott jegyek továbbíthatók más szerverekre
(például ssh-n keresztül).

Amennyiben a kerberos szerver másik gépen van (nálam ugyanaz), akkor szükség van még egy beállításra a fenti hármon
felül. Ez mondja meg, hogy hány másodperccel térhet el a kerberos szerver és az aktuális gép rendszerórája. A név a
\texttt{clockskew}, alapértelmezett értéke 300 (5 perc). Többit lásd \texttt{man 5 krb5.conf}.

A következő szakasz azt mondja meg, hogy hol keresse a kerberos szervereket egy adott realmhoz. Jelen esetben a kdc is
és az kadmin szerver is a gépemen van, a 88-as illetve 749-es porton. Próbáltam localhost-ot megadni, de az nem
működött rendesen. Ugyanitt megadható több további realm is. Ha így nem szerepel, akkor még van egy lehetőség, hogy az
adott realm nevének megfelelő dns bejegyzéseket keres.

Tegyük fel, hogy két kerberos szerverünk van, kdc1 és kdc2. Ekkor a DNS bejegyzések a következők lesznek:

\ut{\_kerberos.\_udp   IN SRV 1 0 88 kdc1\\
  \_kerberos.\_tcp   IN SRV 1 0 88 kdc1\\
  \_kerberos.\_udp   IN SRV 10 0 88 kdc2\\
  \_kerberos.\_tcp   IN SRV 10 0 88 kdc2\\
  \_kerberos-adm.\_tcp IN SRV 1 0 749 kdc1\\
  \_kpasswd.\_udp   IN SRV 1 0 464 kdc1\\
  \_kerberos IN TXT "EXAMPLE.COM"
}

Itt a gépnevek végén nincsen pont, ezért a tartományon belül szerepelnek, pl. a gépem esetén így egészülne ki:\\
\ut{\_kerberos.udp.panthernet. IN SRV 1 0 88 kdc1.panthernet.}

A kdc1 az elsődleges kerberos kiszolgáló, ezért az SRV bejegyzés után álló számot kisebbre kell állítani (1 versus
10). A 88 a kdc standard (krb5-ös) portja (tcp + udp), 749 pedig a kadmin szerveré. A jelszóváltoztatást is és a realm
felügyeletét is a kdc1-en keresztül lehet megtenni.

A krb5.conf fájl következő szakasza a \texttt{[domain\_realm]}, amely azt adja meg, melyik géphez melyik realm
tartozik. Itt lehet tartomány is, ilyenkor ponttal kezdődik, és lehet konkrét gép is. Erre példa a
\texttt{.panthernet} és a \texttt{localhost}. Mindkettőhoz a \texttt{PANTHERNET} realm tartozik.

Ha így nem található meg egy géphez tartozó realm, akkor a rendszer levágja a domain nevet, pl. foo.example.com esetén
ez az example.com, és megnézi, hogy a domain névhez milyen TXT bejegyzés tartozik. A fenti példában az
EXAMPLE.COM. Továbbá megnézi a fent megadott \_kerberos.* bejegyzéseket is, amely révén megtalálja a géphez tartozó
kerberos kiszolgálót.


\subsection{A realm létrehozása}
A(z elsődleges) KDC-n a \texttt{kadmin.local} programmal lehet adminisztrálni a realmot. Ha ezt még nem hoztuk létre,
akkor ilyen hibaüzenetet kapunk:

\ut{\# kadmin.local\\
  Authenticating as principal root/admin@PANTHERNET with password.\\
  kadmin.local: No such file or directory while initializing kadmin.local interface}

\newpage
Hozzuk létre (PANTHERNET helyére értelemszerűen mást írva):

\ut{\# kdb5\_util create -s -r  PANTHERNET\\
  Loading random data\\
  Initializing database '/etc/krb5kdc/principal' for realm 'PANTHERNET',\\
  master key name 'K/M@PANTHERNET'\\
  You will be prompted for the database Master Password.\\
  It is important that you NOT FORGET this password.\\
  Enter KDC database master key:\\
  Re-enter KDC database master key to verify:
}

A realm adatbázisa (fájlja) a következő: \texttt{/etc/krb5kdc/principal}. A fájlt jelszó védi, amit érdemes felírni és
jól elzárt helyen tárolni. Enélkül könnyen visszaélhet valaki a jogosultságokkal. Ha nem ismert a jelszó, akkor nem
lesz mindenre jogunk (nekem helyi gépen nem sikerült helyzetet produkálni, talán csak szerencsém volt).

A legfontosabb bejegyzés a \texttt{K/M@PANTHERNET}, ezt TILOS kiexportálni, mert ennek ismeretében bármilyen
változtatást megtehet a szerveren az, aki csak akar...

Ha meg kell adni a mesterjelszót a \texttt{kadmin.local}-nak, akkor azt a parancs után megadott \texttt{-m} opcióval
tehetjük meg. A program indulás után bekéri a jelszót.

\section{A kerberos bejegyzések létrehozása és listázása}
A \texttt{kadmin.local} kliens programot elindítva elkedhetjük a felhasználókat és a szolgáltatásokat megadni. Első lépésben
házirendeket (policy) érdemes létrehozni, amelynek most különösebb szerepe nincsen, viszont bármikor szükséges lehet a
szabályok szigorítása, és így általánosan kezelhetjük a felhasználókat.

A kliensben van segítség is, mely a \texttt{help} parancs hatására jön elő, illetve egy nem teljesen megadott parancs
is kiírja paraméterezését. A parancsoknak van hosszabb és rövidebb formájuk is.

\ut{\# kadmin.local\\
  Authenticating as principal root/admin@PANTHERNET with password.
  kadmin.local:  \emph{\textbf{addpol}}\\
  add\_policy: parser lost count!\\
  usage; add\_policy [options] policy\\
  \hspace*{1.5cm}options are:\\
  \hspace*{3cm}[-maxlife time] [-minlife time] [-minlength length]\\
  \hspace*{3cm}[-minclasses number] [-history number]\\
  kadmin.local: \emph{\textbf{addpol users}}\\
  kadmin.local: \emph{\textbf{addpol admin}}\\
  kadmin.local: \emph{\textbf{addpol hosts}}
}

A fenti 3 házirenddel szétválaszthatjuk a jogosultságok szerint az ún. \emph{pricipal}-okat (a létrehozott
bejegyzéseket).

Az enyémet (normál ill. admin változatban) így lehet hozzáadni (a parancs 2 alakját használva a 3-ból):

\ut{kadmin.local: \emph{\textbf{ank -policy users panther}}\\
  Enter password for principal "panther@PANTHERNET":\\
  Re-enter password for principal "panther@PANTHERNET":\\
  Principal "panther@PANTHERNET" created.\\
  kadmin.local: \emph{\textbf{addprinc -policy admin panther/admin}}\\
  Enter password for principal "panther/admin@PANTHERNET":\\
  Re-enter password for principal "panther/admin@PANTHERNET":\\
  Principal "panther/admin@PANTHERNET" created.
}


Ha ki szeretnénk listázni az összeset, akkor a következőképp járjunk el (és ilyesmilyen kimenetet kapunk):

\ut{kadmin.local: \emph{\textbf{listprincs}}\\
  K/M@PANTHERNET\\
  \ldots\\
  panther/admin@PANTHERNET
}

Normál felhasználóként indítva más promptot kapunk és másképp kell indítani:

\ut{panther ~ \$ /usr/sbin/kadmin
  Couldn't open log file /var/log/kadmin.log: Permission denied\\
  Couldn't open log file /var/log/kadmin.log: Permission denied\\
  Authenticating as principal panther/admin@PANTHERNET with password.\\
  Password for panther/admin@PANTHERNET:\\
  kadmin:}

Az első két sor elfogadható, hiszen csak a rootnak szabad írnia a log fájlokat. Utána kiírja, hogy a felhasználói fiók
admin változatával próbál csatlakozni, ehhez bekéri a jelszót.

Ekkor sajnos még nincs jogunk listázni sem a principal-okat.

A kdc-t tartalmazó géphez tartozó principal hozzáadása:

\ut{kadmin.local:  \emph{\textbf{addprinc -randkey -policy hosts host/zeratul.panthernet}}\\
  Principal "host/zeratul.panthernet@PANTHERNET" created.\\
  kadmin.local:  \emph{\textbf{ktadd -k /etc/krb5.keytab host/zeratul.panthernet}}\\
  Entry for principal host/zeratul.panthernet with kvno 3, encryption type Triple DES cbc mode with HMAC/sha1 added to
  keytab WRFILE:/etc/krb5.keytab.\\    
  Entry for principal host/zeratul.panthernet with kvno 3, encryption type DES cbc mode with CRC-32 added to keytab
  WRFILE:/etc/krb5.keytab.
}

Most nem kell jelszó, ezért egy véletlen kulcsot generál a program a \texttt{-randkey} opció hatására. Mivel a
kiszolgáló nem adhat meg jelszót, ezért jelszava például a \texttt{/etc/krb5.keytab} fájlban tárolódik, vagy másban, a
lényeg, hogy csak a kiszolgáló felhasználója tudja olvasni. Itt most ebbe a fájlba került.


\subsection{A KDC beállítása}
A  KDC beállítófájlja, amin nem nagyon kell változtatni (/etc/kdc.conf):

\ut{[kdcdefaults]\\
  kdc\_ports = 88,750\newline\newline
  [realms]\\
  PANTHERNET = \{\\
  database\_name = /etc/krb5kdc/principal\\
  admin\_keytab = /etc/krb5kdc/kadm5.keytab\\
  acl\_file = /etc/krb5kdc/kadm5.acl\\
  dict\_file = /etc/krb5kdc/kadm5.dict\\
  key\_stash\_file = /etc/krb5kdc/.k5.PANTHERNET\\
  kadmind\_port = 749\\
  max\_life = 10h 0m 0s\\
  max\_renewable\_life = 7d 0h 0m 0s\\
  master\_key\_type = des3-hmac-sha1\\
  supported\_enctypes = des3-hmac-sha1:normal des-cbc-crc:normal\\
  \}}

Egy KDC több realmot is kiszolgálhat. Amennyiben automatikusan szeretnénk elindítani, a mester jelszó megadása
nélkül, akkor a \texttt{key\_stash\_file} opcióban megadott fájlra szükségünk van, ellenkező esetben töröljük le. Az
automatikus indítás a biztonság rovására megy, mert a fájl megszerzésével vissza lehet élni.

A \texttt{/etc/krb5kdc/kadm5.acl} fájl tartalmazza a jogosultságokat, jelen esetben:

\ut{*/admin@PANTHERNET      *}

vagyis az összes adminnak mindenhez van joga.

A \texttt{/etc/krb5kdc/kadm5.keytab} az admin keytab (kulcsokat tartalmazó fájl), ide két principal-t kell betenni:

\ut{kadmin.local:  \emph{\textbf{ktadd -k /etc/krb5kdc/kadm5.keytab kadmin/admin kadmin/changepw}}\\
  Entry for principal kadmin/admin with kvno 3, encryption type Triple DES cbc mode with HMAC/sha1 added to keytab WRFILE:/etc/krb5kdc/kadm5.keytab.\\
  Entry for principal kadmin/admin with kvno 3, encryption type DES cbc mode with CRC-32 added to keytab WRFILE:/etc/krb5kdc/kadm5.keytab.\\
  Entry for principal kadmin/changepw with kvno 4, encryption type Triple DES cbc mode with HMAC/sha1 added to keytab WRFILE:/etc/krb5kdc/kadm5.keytab.\\
  Entry for principal kadmin/changepw with kvno 4, encryption type DES cbc mode with CRC-32 added to keytab WRFILE:/etc/krb5kdc/kadm5.keytab.\\
}

\section{Szolgáltatások hozzáadása}
Tegyük fel, hogy a cyrus szervert szeretnénk beüzemelni. Neki külön keytab fájl kell, ezt a következőképpen tehetjük
meg (kadmin vagy kadmin.local is jó):

\ut{admin.local:  \emph{\textbf{addprinc -randkey -policy hosts imap/zeratul.panthernet}}\\
  Principal "imap/zeratul.panthernet@PANTHERNET" created.\\
  kadmin.local:  \emph{\textbf{ktadd -k /etc/krb5.keytab.cyrus imap/zeratul.panthernet}}\\
  Entry for principal imap/zeratul.panthernet with kvno 3, encryption type Triple DES cbc mode with HMAC/sha1 added to
  keytab WRFILE:/etc/krb5.keytab.cyrus.\\    
  Entry for principal imap/zeratul.panthernet with kvno 3, encryption type DES cbc mode with CRC-32 added to keytab
  WRFILE:/etc/krb5.keytab.cyrus.\\    
}

Bármely további szolgáltatást hasonlóképpen, \emph{szolgáltatásnév/szervernév}  formában megadhatunk. Itt a
/etc/krb5.keytab.cyrus fájl a cyrus tulajdonában lévő, csak általa olvasható (és esetleg írható) fájl.


\section{Jegyek igénylése, kezelése}
Általában bejelentkezéskor kapunk egy jegyet, ahogy fentebb elhangzott, a \emph{tgt}-t, azonban néha szükség lehet
arra, hogy menet közben ezt eldobjuk, vagy éppen másik realmból szeretnénk jegyet kapni.

Például bejelentkeztem az \texttt{INF.ELTE.HU} tartományba, és a host ticketeknek megfelelően a 3 gépre:

\ut{\$ klist\\
  Ticket cache: FILE:/tmp/krb5cc\_1000\\
  Default principal: panther@INF.ELTE.HU\\
  \\
  Valid starting\ \ \ \ \ Expires\ \ \ \ \ \ \ \ \ \ \ \ Service principal\\
  07/30/06 23:59:19\  07/31/06 09:59:30\ \ krbtgt/INF.ELTE.HU@INF.ELTE.HU\\
  \ \ \ \ \ \ \ \ renew until 07/31/06 23:59:19\\
  07/30/06 23:59:54\ \ 07/31/06 09:59:30\ \ host/valerie.inf.elte.hu@INF\ldots\\
  \ \ \ \ \ \ \ \ renew until 07/31/06 23:59:19\\
  07/30/06 23:59:54\ \ 07/31/06 09:59:30\ \ host/panda.inf.elte.hu@INF\ldots\\
  \ \ \ \ \ \ \ \ renew until 07/31/06 23:59:19\\
  07/30/06 23:59:54\ \ 07/31/06 09:59:30\ \ host/pandora.inf.elte.hu@INF\ldots\\
  \ \ \ \ \ \ \ \ renew until 07/31/06 23:59:19
}

Itt a jegy a \texttt{/tmp/krb5cc\_1000} fájlban van, amelyben az alapértelmezett (principal) a
\texttt{panther@INF.ELTE.HU}.

Ennek eldobása:

\ut{kdestroy}

Ha nincsen ``default principal'', akkor a \texttt{/etc/krb5.conf} alapértelmezett realm-ját veszi alapul, ellenkező
esetben a principalt újítja meg:

\ut{\$ kinit\\
  Password for panther@INF.ELTE.HU: }

A \texttt{kinit} egyik legfontosabb opciója a \texttt{-f} és a \texttt{-F}, az előbbivel lehet \emph{továbbítható
  (forwardable)} jegyet igényelni, utóbbi pedig letiltja ezt. Ha továbbítható a jegy, akkor jelszókérés nélkül be lehet
jelentkezni másik gépre is, ugyanis a jegy azonosít minket.


\section{A pam beállítása}
Ahhoz, hogy bejelentkezéskor a kerberosban tárolt jelszavunkat használjuk, be kell állítani a PAM-ban a kerberos
használatát.

A PAM a ``Pluggable Authentication Modules for Linux'' rövidítése, olyan hitelesítő modulokból áll, amelyek szabadon
cserélhetőek. A változtatások azonnal érvénybe lépnek, a PAM-ot használó alkalmazás újraindítása nélkül is. A
beállításokat a \texttt{/etc/pam.d} könyvtár tartalmazza.

A Gentoo Linuxon az általánosan használt rész egy külön fájlban,\\ a \texttt{/etc/pam.d/system-auth}-ban van:

\ut{\begin{tabular}{llll}%{\hspace{0.4cm}l\hspace{0.4cm}l\hspace{0.4cm}l\hspace{0.2cm}l}
    auth      & required    & pam\_env.so&\\
    auth      & sufficient  & pam\_unix.so &likeauth nullok\\
    auth      & sufficient  & pam\_krb5.so &use\_first\_pass forwardable\\
    auth      & required    & pam\_deny.so&\\\\
    account   & required    & pam\_unix.so&\\\\
    password  & required    & pam\_cracklib.so &difok=2 minlen=8 dcredit=2\textbackslash\\
    &&&\hspace*{1cm}ocredit=2 retry=3\\
    password  & sufficient &  pam\_unix.so& nullok md5 shadow use\_authtok\\
    password  & sufficient &  pam\_krb5.so &use\_first\_pass\\
    password  & required   &  pam\_deny.so&\\\\
    session   & required  &   pam\_limits.so&\\
    session    &sufficient &  pam\_unix.so&\\
    session    &sufficient  & pam\_krb5.so&\\
    session    &required    & pam\_deny.so&\\
    session    &required    &pam\_unix.so&
  \end{tabular}
}

Ez egy tipikusan hibás fájl, ami viszont jól szerepel benne: a \texttt{pam\_krb5}-öt ugyanúgy kell kezelni, mint a
\texttt{pam\_unix}-ot. Az előbbi az első jelszót használja, ha nem sikerülne a \texttt{pam\_unix} jelszóellenőrzése,
valamint továbbítható jegyet kér.

Hogyan lehet használni a fenti fájlt? Például így:

\ut{\begin{tabular}{lll}
    auth      & required    & pam\_nologin.so\\
    auth      & include     & system-auth\\
    account   & include     & system-auth\\
    session   & include     & system-auth
  \end{tabular}
}

% Local Variables:
% fill-column: 120
% TeX-master: t
% End:

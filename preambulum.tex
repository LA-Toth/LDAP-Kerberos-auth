% vim:tw=120:sw=2:enc=utf-8

% csomag opciók
\def\magyarOptions{frenchspacing=yes,activeprefix=grave}

% csomagok
\usepackage[utf8]{inputenc}
\usepackage{t1enc}
\usepackage[english,magyar]{babel}
\usepackage[dvips]{graphicx}
\usepackage{graphics}
\usepackage{graphicx}
\usepackage{fancyvrb}
%\usepackage{amsrefs}
\usepackage{amsmath}
\usepackage{alltt}
\usepackage{verbatim}
\usepackage{booktabs}
\usepackage{paralist}
\usepackage{fancyhdr}
\usepackage{xspace}

% makrók
\makeatletter
\def\Ifnopdflatex{\csname @ifundefined\endcsname{pdfcompresslevel}}
\makeatother

% méretek
%\addtolength{\voffset}{-1cm}
\addtolength{\textheight}{2cm}
\addtolength{\hoffset}{-1.2cm}
\addtolength{\textwidth}{2.4cm}
\setlength{\headheight}{16pt}

\renewcommand{\headrulewidth}{0.5pt}
\renewcommand{\footrulewidth}{0pt}

\addtolength{\oddsidemargin}{9mm}
\addtolength{\evensidemargin}{-7mm}

\pagestyle{fancy}
\fancyhf{}
\fancyhead[LE,RO]{\sffamily\thepage}%
\fancyhead[LO]{\sffamily\sc\nouppercase{\rightmark}}%
\fancyhead[RE]{\sffamily\sc\nouppercase{\leftmark}}%


\fancypagestyle{plain}{\fancyhf{}\renewcommand{\headrulewidth}{0pt}}

\newcounter{fejezet}
\makeatletter
\renewcommand{\chaptermark}[1]{\markboth{\ifnum \c@fejezet = 1
    \ifnum \thechapter > 0
          \thechapter.\ \chaptername.\ %
	     \fi
 \fi#1}{}
}

\renewcommand{\sectionmark}[1]{\markright{\thesection.\ #1}}
\makeatother


\newcommand{\chaptern}[1]{
\setcounter{fejezet}{0}%
\chapter*{#1}%
\addcontentsline{toc}{chapter}{#1}%
\chaptermark{#1}%
}

\newcommand{\chaptery}[1]{
\setcounter{fejezet}{1}%
\chapter{#1}%
}


\makeatletter
\def\cleardoublepage{\clearpage\if@twoside \ifodd\c@page\else
\hbox{}
\thispagestyle{empty}
\newpage
\if@twocolumn\hbox{}\nepage\fi\fi\fi}
\makeatother

\clearpage{\thispagestyle{empty}\cleardoublepage}
\makeatletter
\newcommand{\elv}{\leavevmode\nobreak-\hskip\z@skip}
\makeatother

%%%
% Magyar tipográfiai előírások
%%%

\baselineskip=0pt
\parskip=0pt


%%%%%%

\newcommand{\fontos}[1]{\texttt{#1}}
\newcommand{\ut}[1]{\vspace{5mm}\fbox{\parbox{130mm}{\texttt{\noindent{#1}}}}\vspace{5mm}}
\newcommand*{\filename}[1]{\texttt{#1}}
\newcommand*{\behuzas}[1][1]{\hspace*{#1em}}
\newcommand*{\beh}[1][1]{\hspace*{#1em}}
\newcommand*{\tgt}{\textgreater}


\newcommand{\xml}{\textsc{xml}\xspace}


%%%% Környezetek
\newcommand{\listazjbetu}{
  \renewcommand{\theenumi}{\alph{enumi}}
  \renewcommand{\labelenumi}{(\theenumi)}
}
\newcommand{\listazjromai}{
  \renewcommand{\theenumi}{\alph{enumi}}
  \renewcommand{\labelenumi}{(\theenumi)}
}
\newcommand{\listabetu}{
  \renewcommand{\theenumi}{\alph{enumi}}
  \renewcommand{\labelenumi}{\theenumi}
}
\newcommand{\listaszamkor}{
  \renewcommand{\theenumi}{\alph{enumi}}
  \renewcommand{\labelenumi}{\theenumi$^\circ$}
}
\newenvironment{enumzjromai}{\listazjromai\begin{enumerate}}{\end{enumerate}}
\newenvironment{enumzjbetu}{\listazjbetu\begin{enumerate}}{\end{enumerate}}

\newenvironment{enumzjr}{\begin{enumzjromai}}{\end{enumzjromai}}
\newenvironment{enumzjb}{\begin{enumzjbetu}}{\end{enumzjbetu}}




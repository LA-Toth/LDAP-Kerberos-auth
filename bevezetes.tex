\chapter{Bevezetés}

Bár ma a számítógépek és az Internet elterjedésével egyre több helyen épülnek ki sokfelhasználós, számos szolgáltatást
nyújtó rendszerek, bonyolult kezdetben a felhasználók hitelesítésére elegendő volt egy egyszerű szövegfájl, amely
tartalmazta a felhasználó adatait, valamint a jelszavát is. Mindenkinek joga volt olvasni ezt a fájlt, a jelszavak
tárolása ezáltal nem volt biztonságos, de az akkori gépigénye miatt nem volt érdemes megpróbálni visszafejteni Ezért a
jelszavakat később már külön állományban tárolták, amit a gépen egyetlen felhasználó, a root tudott mind írni, mind
olvasni.

A biztonsági problémát sikerült elhárítani, azonban két további probléma változatlanul fennállt, az egyik az, hogy
néhány ezer vagy különösen több tízezer felhasználó esetén kezelhetetlenné vált mind módosítás, mind keresés esetén. A
másik akkor merült fel, ha egyazon felhasználónak több gépre is van bejelentkezési joga, ennek megfelelően ma sokszor
egy-egy központi szerveren létrehozzák a felhasználó fiókját (ahol a fájljai, könyvtárai, tehát lényegében minden adata
tárolódik), így számos egyéb gépen (például laborokban) azonos körülmények között folyhat a munka. A felhasználókat
tároló fájlt minden gépre át kellene másolni, módosításokat követni, ami nagyon nehézkes, a probléma ugyanaz, mint  mint
alig több, mint egy évtizeddel korábban gépek azonosításával, amikor sok gép volt már, és az őket azonosító számokat,
valamint a könnyebb megjegyezhetőség végett bevezetett beszédes neveket összekapcsoló állományokat körülményes volt
karbantartani. Ezért bevezették a névkiszolgálók rendszerét, s csak egy-egy gépen kell a módosításokat elvégezni.

A hitelesítésnél ugyanez a megoldás, egy központi gépen tárolható a felhasználó összes adata, jelszava. Többféle
megoldás is szóba kerülhet, például adatbázisban (MySQL, Oracle, stb.), vagy épp egy hierarchikus rendszerben
(pl. \textsc{ldap}-ban) történő tárolás. Az \textsc{ldap} protokoll egy könnyen kezelhető, gyors megoldást nyújt, ezt
valósítja meg az OpenLDAP szerver, valamint a Microsoft Active Directory (\textsc{AD}) vezérlői is. 

Az \textsc{ldap} adatbázisban is tárolható a jelszó, azonban nem túl biztonságos, ráadásul minden egyes szolgáltatás
igénybevételekor újra be kell írni a jelszót. Mind a biztonság, mind a kényelem szempontjából előnyösebb a Kerberos
használata, amit az \textsc{mit} mit-krb5 programja, valamint a Heimdal valósít meg a nyílt forráskódú rendszerek
esetén, és például Microsoft Windows AD tartományvezérlői is tartalmazzák. A jegyzetben a mit-krb5-ről lesz csak szó.

Az \textsc{ldap} és a Kerberos segítségével bejelentkezés után többet nem kell beírni a jelszót, azonnal be lehet
bejelentkezni bármely másik gépre is, valamint email-eket is el lehet olvasni, egészen pontosan bármely Kerberos-t
támogató szolgáltatás azonnal, biztonságosan, és újbóli jelszó begépelés nélkül igénybe vehető. Ezek közül a
levelezőszerverek (jelen esetben Postfix és Cyrus), és levelezőkliensek, valamint a bejelentkezés beállítását tekintjük
csak meg, mivel a többi szolgáltatás ezeknél jóval ritkábban fordul elő.

Az itt bemutatott rendszer GNU/Linux alapokon nyugszik, a jegyzet elsősorban a Gentoo Linuxon történő beállításokat
mutatja be, azonban a többi terjesztésen (disztribúción) is hasonlóan használható.

A továbbiakban a következő szolgáltatásokat vizsgáljuk: 
OpenLDAP,
mit-krb5 (Kerberos),
Tanúsítvány-kiszolgáló (Certificate Authority),
PAM,
Cyrus saslauthd,
OpenSSH,
Postfix,
Amavisd-new,
ClamAV,
SpamAssassin,
valamint \textsc{ssl}/\textsc{tls} titkosítás az előző programok esetén. A beállításokat lépésről lépésre nézzük át, először mindig
azt, hogy önállóan, a többi nélkül hogyan lehet beállítani, majd pedig a rendszerbe történő integrálásukat.

Egy nagyobb, sok számítógépből álló rendszerben a legtöbb szolgáltatás önálló gépen, vagy gépeken fut, ezért
elengedhetetlen, hogy a hálózati kommunikáció titkosítva történjen. Néhány esetben enélkül nem is lehet igénybe venni a
szolgáltatást (\textsc{ssh}, Kerberos), azonban több is megengedi a titkosítatlan használatot, amely túl nagy
kockázattal járhat (ilyenek a levelezőszerverek, de maga az \textsc{ldap} kiszolgáló is). A titkosítást \textsc{ssl}
(Secure Socket Layer) feletti kommunikáció teszi lehetővé, amely biztonsága egyrészt a titkosítás erőssége (mennyire
könnyű visszafejteni), másrészt a tanúsítványok használata adja, amelyekkel ellenőrizhető, hogy kivel történik a
kommunikáció.

  
% Local Variables:
% fill-column: 120
% TeX-master: t
% End:
